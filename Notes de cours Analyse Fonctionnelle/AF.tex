\documentclass[a4paper,10pt,preprint,aps,tightenlines,showpacs,superscriptaddress]{revtex4}	%Rajouter a4paper,12pt suites à relecture des instructions
\usepackage[top=2cm, bottom=2cm, left=1cm, right=1cm]{geometry}	% new package
\usepackage[dvips]{graphicx}
\usepackage{dcolumn}
\usepackage{epsfig}
\usepackage{color}
\usepackage{bm}
\usepackage{gensymb}
\usepackage{amsmath}
\usepackage{amsthm}
\usepackage{appendix}
\usepackage[english]{babel}
\usepackage[T1]{fontenc}
\usepackage{fancyhdr}
\usepackage{amsfonts}
\usepackage{comment}

\def\bra#1{\mathinner{\langle{#1}|}}
\def\ket#1{\mathinner{|{#1}\rangle}}
\def\braket#1{\mathinner{\langle{#1}\rangle}}

\graphicspath{{tpillustrations/}}

\def\slash#1{\not\!#1}
\def\slashb#1{\not\!\!#1}
\def\delsla{\not\!\partial}

\newcommand{\red}[1]{\textcolor{red}{#1}}
\newcommand{\blue}[1]{\textcolor{blue}{#1}}
\newcommand{\green}[1]{\textcolor{green}{#1}}
\newcommand{\magenta}[1]{\textcolor{magenta}{#1}}

\newcommand{\sla}[1]{\not\! #1}
\def\ohalf{{\textstyle{1\over 2}}}
\def\half{{\textstyle{1\over 2}}}
\def\vqhalf{{\textstyle{\vec{Q}\over 2}}}
\def\qhalf{{\textstyle{Q\over 2}}}
\def\osix{{\textstyle{1\over 6}}}
\def\vqsix{{\textstyle{\vec{Q}\over 6}}}
\def\thalf{{\textstyle{3\over 2}}}
\def\fourth{{\textstyle{1\over 4}}}
\def\tfor{{\textstyle{3\over 4}}}

\newcommand{\inp}[2]{\langle{#1}|{#2}\rangle}
\newcommand{\vev}[2]{\langle{#1}\rangle}
\newcommand{\slas}[1]{\not\!{#1}}
\def\ohalf{{\textstyle{1\over 2}}}
\def\thalf{{\textstyle{3\over 2}}}
\def\fhalf{{\textstyle{5\over 2}}}
\def\shalf{{\textstyle{7\over 2}}}
\def\nhalf{{\textstyle{9\over 2}}}


\newcommand\Tr{\,{\rm Tr}\,}
\newcommand\re{\Re\mbox{e}}
\newcommand\im{\Im\mbox{m}}
\newcommand{\beq}{\begin{equation}}
\newcommand{\eeq}{\end{equation}}

\pagestyle{fancy}
\lhead{\scriptsize{ÉCOLE POLYTECHNIQUE FÉDÉRALE DE LAUSANNE\\
3ème année de Physique\\
Baptiste CLAUDON}}
\rhead{\empty}
\renewcommand{\headrulewidth}{0pt}

\theoremstyle{definition}
\newtheorem{Def}{Définition}
\newcommand{\begdef}{\begin{Def}}
%\newcommand{\enddef}{\end{Def}}
\newtheorem{Prop}{Proposition}
\newcommand{\begpro}{\begin{Prop}}
%\newcommand{\endpro}{\end{Prop}}
\newtheorem{The}{Théorème}
\newcommand{\begthe}{\begin{The}}
%\newcommand{\endthe}{\end{The}}
\newtheorem*{Pre}{Preuve}
\newcommand{\begpre}{\begin{Pre}}
\newtheorem{Ex}{Exemple}
\newtheorem{Post}{Postulat}
\newtheorem{Cor}{Corrolaire}
\newcommand{\begcor}{\begin{Cor}}
\newtheorem{Lem}{Lemme}

\begin{document}

\begin{titlepage}
   \vspace*{\stretch{1.0}}
   \begin{center}
      \Huge\textbf{Analyse Fonctionnelle\\pour Physiciens}\\
      %\Large{Introduction aux Ondes\\ et à la Mécanique Quantique}\\ 
      \text{ }\newline
      \large{Baptiste Claudon}\\
      \today
   \end{center}
   \vspace*{\stretch{2.0}}
\end{titlepage}

\tableofcontents\newpage
\newpage
\part{Espaces Fonctionnels}
\section{Le théorème de Stone-Weierstrass}
\begthe\textbf{Théorème de Diniz}
Soit $(f_n)$ une suite de fonctions réelles et continues définies sur un compact $K\subset\mathbb R^n$ et convergent simplement et de manière monotone vers $f\in C(K,\mathbb R)$. Alors cette suite converge uniformément vers $f$.\label{diniz}
\end{The}
\begpre
Choisir, sans perte de généralité que $(f_n)$ est décroissante et converge simplement vers $0$. Soit $\epsilon>0$. Poser pour $n\in\mathbb N$ : 
\beq
V_n=\{x\in K:f_n(x)<\epsilon\}
\eeq
Par continuité des fonctions de la suite, tous ces ensembles sont des ouverts. Puisque la suite tend vers $0$, on a : 
\beq
K=\bigcup_{n\in\mathbb N}V_n
\eeq
$K$ étant compact, il existe un nombre $F\in\mathbb N$ tel que
\beq
K=\bigcup_{n=0}^FV_n=K
\eeq
Puisque la suite est monotone décroissante, on a que $m<n$ implique $V_m\subseteq V_n$, donc $V_F=K$.
\qed
\end{Pre}

\begdef
Soit $F$ une famille de fonctions définies sur un ensemble $X\subset\mathbb R^n$. On dit que $F$ sépare $X$ si :
\beq
\forall x,y\in X, x\neq y,\exists f\in F:f(x)\neq f(y)
\eeq
\end{Def}
\begdef
On dit que $F$ ne s'annule pas sur $X$ si :
\beq
\forall x\in X\exists f\in F:f(x)\neq0
\eeq
\end{Def}
\begdef
Si $B$ est un sous-ensemble d'une $\mathbb K$-algèbre $A$, alors la $\mathbb K$-algèbre engendrée par $B$, $\mathcal A_\mathbb K(B)$ est la plus petite $\mathbb K$-algèbre contenant $B$.
\end{Def}
\begthe\textbf{Théorème de Stone-Weierstrass} Soit $X\subset\mathbb R^n$ un ensemble compact et soit $F\subseteq C(X,\mathbb R)$ une famille de fonctions qui sépare $X$ et qui ne s'annule pas sur $X$. Alors l'algèbre réelle $\mathcal A_\mathbb K(F)$ engendrée par $F$ est uniformément dense dans $C(X,\mathbb R)$ : 
\beq
\overline{\mathcal A_\mathbb K(F)}^{||\cdot||_\infty}=C(X,\mathbb R)
\eeq
\label{sw}
\end{The}

\begpre 
%%	ÉTAPE 1
\begin{Lem}
Il existe une suite $(P_n)$ de polynômes sur $[-1,1]$ qui converge uniformément vers $x$.\label{step1}
\end{Lem}
Définir la suite de polynômes par $P_0=0$ et :
\begin{eqnarray}
\forall n\geq0:P_{n+1}(x)=P_n(x)+\frac12(x^2-P_n(x)^2)
\end{eqnarray}
On déduit alors par récurrence que $\forall n\in\mathbb N,\forall x\in[-1,1] :0\leq P_n(x)\leq|x|$. Alors, on déduit que la suite $(P_n)$ est croissante. Puisque qu'elle est majorée elle doit converger. La limite doit être point fixe de l'application donc est $x\mapsto|x|$. Par le théorème de Diniz \ref{diniz}, la convergence est uniforme.
%%	ÉTAPE 2
\begin{Lem}
Si $f,g_i\in\overline{\mathcal A(F)}$, $i=1,...,n$, alors $|f|, \min_{i=1,...,n}g_i,\max_{i=1,...,n}g_i$ appartiennent aussi à $\overline{\mathcal A(F)}$. \label{step2}
\end{Lem}
Si $f\neq0$, poser $h=\frac{f}{||f||}$. Noter $h\in\overline{\mathcal A(F)}$ et a son image dans $[-1,1]$. Par le lemme \ref{step1}, il existe un polynôme $P$ tel que $P(h)$ converge uniformément vers $\frac{|f|}{||f||}$. Pour $g_1,g_2\in\overline{\mathcal A(F)}$, utiliser :
\beq
\min\{g_1,g_2\}=\frac{f+g+|f-g|}2
\eeq
et
\beq
\max\{g_1,g_2\}=\frac{f+g-|f-g|}2
\eeq
Conclure alors récursivement.
%%	ÉTAPE 3
\begin{Lem}
Pour tous $x,y\in X$ et $x\neq y$, et pour tout couple de réels $\alpha,\beta$, il existe un élément $f\in\mathcal A(F)$ tel que $f(x)=\alpha$ et $f(y)=\beta$.\label{step3}
\end{Lem}
Avec les hypothèses du théorème, pour $z=x,y$, il existe $h_z\in F:h_z(x)\neq0$. Il existe également $g\in F$ tel que $g(x)\neq g(y)$. La fonction :
\beq
f:X\to\mathbb R,t\mapsto \alpha\frac{g(t)-g(y)}{g(x)-g(y)}\frac{h_x(t)}{h_x(x)}+\beta\frac{g(t)-g(x)}{g(y)-g(x)}\frac{h_y(t)}{h_y(y)}
\eeq
satisfait les hypothèses du lemme.
%%	ÉTAPE 4
\begin{Lem}
Soient $f\in C(X,\mathbb R), x_0\in X$ et $\epsilon>0$. Alors il existe $g_{x_0,\epsilon}\in\overline{\mathcal A(F)}$ :
\beq
g_{x_0,\epsilon}(x_0)=f(x_0)\text{ et }g_{x_0,\epsilon}<f+\epsilon
\eeq\label{step4}
\end{Lem}
Grâce au lemme \ref{step3}, pour chaque $y\in X\backslash\{x_0\}$, on peut choisir $h_y\in\mathcal A(F)$ telle que $f(y)=h_y(y)$ et $f(x_0)=f_y(x_0)$. Puisque $h_y$ et $f$ sont continues, l'ensemble :
\beq U_y=\{x\in X:h_y(x)<f(x)+\epsilon\}
\eeq
est un ouvert. Puisque $\forall y\in X$, $y\in U_y$, on a (l'inclusion réciproque étant explicite) :
\beq
X=\bigcup_{y\in X} U_y
\eeq
Comme $X$ est compact, il existe une sous-collection d'ouvert finie $\{U_{y_i}\}_{1\leq i\leq p}$ telle que :
\beq
X=\bigcup_{i=1}^p U_y
\eeq
Poser alors $g_{x_0,\epsilon}=\max_{i=1,...,p}h_{y_i}$. Cette fonction satisfait les hypothèses du lemme et appartient à l'algèbre engendrée par le lemme \ref{step2}.
%%	ÉTAPE 5
\begin{Lem}
Soient $f\in C(X,\mathbb R)$ et $\epsilon>0$. Il existe une fonction $g\in\overline{\mathcal A(F)}$ telle que $f-\epsilon\leq g\leq f+\epsilon$.
\end{Lem}
Par le lemme \ref{step4}, on peut choisir pour tout $x\in X$ une fonction $g_{x,\epsilon}\in\overline{\mathcal A(F)}$ telle que $g_{x,\epsilon}(x)=f(x)$ et $g_{x,\epsilon}<f+\epsilon$. Pour chaque $x\in V$, définir : 
\beq V_x=\{z\in X:f(x)-\epsilon<g_{x,\epsilon}(z)\}
\eeq
Puisque $g_{x,\epsilon}$ est définie comme le maximum de plusieurs fonctions continues, elle est continue. $V_x$ est donc ouvert pour chaque $x\in X$. Procédant comme auparavant, remarquer que l'union des membres de la famille d'ouverts vaut $X$, en extraire une sous-famille finie. Définir cette fois $g$ comme le minimum des fonctions sélectionnées. Par le lemme \ref{step2}, $g\in\overline{\mathcal A(F)}$. Elle vérifie de plus les propriétés recherchées par le lemme, et plus généralement par le théorème de Stone-Weierstrass.
\qed
\end{Pre}

\begcor
Soit $X\subset\mathbb R^n$ un ensemble compact et $F\subseteq C(X,\mathbb C)$ une famille de fonction qui sépare $X$, invariante sous conjugaison complexe et qui ne s'annule pas sur $X$. Alors l'algèbre complexe $\mathcal A_\mathbb C(F)$ engendrée par $F$ est uniformément dense dans $C(X,\mathbb C)$, c'est-à-dire $f\in\overline{\mathcal A(F)}$.
\label{cor1}

\end{Cor}
\begpre
On a $F=F^*$ car : 
\beq F^*\subseteq F=(F^*)^*\subseteq F^*\eeq Comme $F$ sépare $X$ et ne s'annule pas sur $X$, $G=(F+F^*)\cup i(F-F^*)$ ne s'annule pas sur $X$ non plus et sépare aussi $X$. Or $F\subseteq C(X,\mathbb R)$ et par le théorème de Stone-Weierstrass \ref{sw}, $C(X,\mathbb R)=\overline{\mathcal A_\mathbb R(G)}$. Comme $C(X,\mathbb C)=C(X,\mathbb R)+iC(X,\mathbb R)$ et que $\overline{\mathcal A_\mathbb R(G)},i\overline{\mathcal A_\mathbb R(G)}\subset \overline{\mathcal A_\mathbb C(G)}$, on a que $C(X,\mathbb C)=\overline{\mathcal A_\mathbb C(G)}$. Or : $\mathcal A_\mathbb C(G)=\mathcal A_\mathbb C(F)$.
\qed
\end{Pre}

\begcor
Soit $X\subset\mathbb R$ un ensemble compact. L'ensemble $\mathbb C[ X]$ est uniformément dense dans $C(X,\mathbb C)$.
\end{Cor}
\begpre
$\mathbb C[ X]=\mathcal A_\mathbb C(\{1,id_X\})
$ vérifie les hypothèses du corollaire \ref{cor1}.
\qed
\end{Pre}

\begcor
Soit $I=[a,b]$ un intervalle fermé de $\mathbb R$. L'algèbre engendrée sur les complexes par $F$ défini comme : 
\beq
F=\left\{e^{2\pi ni\frac{x-a}{b-a}},x\in I, n\in\mathbb N\right\}
\eeq
est uniformément dense dans $V=\{f:f\in C([a,b],\mathbb C),f(a)=f(b)\}$. 
\end{Cor}
\begpre
La fonction $\varphi$ définie par :
\beq
\varphi : [a,b]\to\partial B_1(0), x\mapsto e^{2\pi ni\frac{x-a}{b-a}}
\eeq
induit un homéomorphisme isométrique $\Phi:C(\partial B_1(0),\mathbb C)\to V, f\mapsto f\circ\varphi$. Or, $C(\partial B_1(0),\mathbb C)=\overline{\mathcal A_\mathbb C(\{1,z\mapsto z,z\mapsto z^*\})}$ puisque $\{1,z\mapsto z,z\mapsto z^*\}$ satisfait les hypothèses du corollaire \ref{cor1} et $F=\Phi|_{\mathcal A_\mathbb C(\{1,z\mapsto z,z\mapsto z^*\})}$.
\qed
\end{Pre}

\begin{comment}

\section{Complétion des espaces $L^p$}
\begdef
Une mesure $\mu$ définie sur une $\sigma$-algèbre $\Sigma\subseteq\mathcal P(\mathbb R^n)$ borélienne est dite intérieurement régulière si :
\beq
\forall E\in\Sigma, \mu(E)=\sup\{\mu(K):K \text{ compact et } K\subset E\}
\eeq
Elle est dite extérieurement-régulière si
\beq
\forall E\in\Sigma, \mu(E)=\inf\{\mu(V):V \text{ ouvert et } E\subset V\}
\eeq
Elle est enfin régulière si elle est simultanément extérieurement et intérieurement régulière, et localement finie si :
\beq
\forall x\in\mathbb R^n,\exists \text{ un ouvert }U\in\Sigma : x\in U\text{ et } \mu(U)<\infty
\eeq
\end{Def}

\begin{The}
Soit $\mu$ une mesure régulière et localement finie, $f\in L^p(\mathbb R^n,\mu)$ pour $1\leq p<\infty$ et $\epsilon>0$. Alors :
\beq
\exists \varphi\in C_c(\mathbb R^n):||f-\varphi||_p<\epsilon
\eeq
\end{The}

\begpre
Trouvons $\varphi\in C_c(\mathbb R^n,\mathbb R^+)$ pour une fonction $f\in L^p(\mathbb R^n,\mu)$ positive. Poser pour $m\in\mathbb N$ :
\beq
f_m:x\mapsto \chi_{B_m(0)}(x)\min\{f(x),m\}
\eeq
Alors, pour chaque $m\geq0$, $f_m\in L^1(\mathbb R^n,\mu)$. Puisque $|f-f_m|^p\leq|f|^p$, le théorème de la convergence dominée implique que $(f_m)$ converge vers $f$ dans la norme $||\cdot||_p$. Il existe donc $k\in\mathbb N$ tel que $m\geq k$ implique $||f_k-f||_p<\epsilon/2$. Par définition de l'intégrale de Lebesgue, et puisque $f_m\in L^1(\mathbb R^n,\mu)$, on peut approcher $f_m$ par une fonction étagée simple positive, celle-ci se laissant approcher par une fonction $\varphi\in C_c(\mathbb R^n,\mathbb R_+)$, telle que $||f_m-\varphi||_1<\frac{\epsilon^p}{2^pm^{p-1}}$. Puisque $f_m$ est majorée par $m$, on peut le supposer aussi pour $\varphi$. On a alors $|f_m(x)-\varphi(x)|\leq m$, et $|f_m-\varphi|^p\leq m^{p-1}|f_m-\varphi|$. Par conséquent,
\beq
||f_m-\varphi||_p^p\leq m^{p-1}||f_m-\varphi||_1\leq\left(\frac\epsilon2\right)^p
\eeq
C'est-à-dire $||f-\varphi||_p<\epsilon$.
\qed\end{Pre}

\begthe
Si $\mu$ est une mesure régulière et localement finie, l'espace de Hilbert $L^2(\mathbb R^n,\mu)$ est séparable.
\end{The}

\begpre
Soit $f\in C_c(\mathbb R^n)$. Des fonctions qui sont des combinaisons linéaires du type $\sum_{k=1}^m\alpha_k\chi_{E_k}$ pour $E_k=\times_{j=1}^n ]a_j,b_j]$ approchent $f$ uniformément dans $C_c(\mathbb R^n)$. Par conséquent, pour une mesure régulière et localement finie, les fonctions de ce type approchent $f$ dans $L^2(\mathbb R^n,\mu)$. On pourrait donc choisir les $\alpha_k,a_k,b_k$ rationnels pour approcher $f$ par une famille dénombrable.
\qed
\end{Pre}
\end{comment}
\section{Systèmes complets et orthonormés}
\begdef 
Soit $\mathcal H$ un $\mathbb K$-espace vectoriel. Une famille $F\subseteq \mathcal H$ est dite complète si l'ensemble $\text{Vect}(F)$ des combinaisons linéaires d'éléments de $F$ est un ensemble dense de $\mathcal H$.
\end{Def}

\begthe Soit $\mathcal H$ un espace de Hilbert sur $\mathbb K$ séparable. Soit $B=\{\varphi_n\}_{n\in I}$ une famille de vecteurs othonormée. Alors $|B|\leq|\mathbb N|$. \end{The}

\begpre
Puisque $\mathcal H$ est séparable, il existe une famille $D=\{\psi_n\}_{n\in \mathbb N}\subset \mathcal H$ dense dans $\mathcal H$. Il existe alors pour chaque $\varphi_k\in B$ un élément de $\psi_k\in D$ tel que $||\varphi_k-\psi_k||\leq 1/\sqrt2$. Si $\varphi_k\neq\varphi_l$, alors par orthonormalité $||\varphi_k-\varphi_l||^2=2$ est par conséquent, $||\varphi_k-\varphi_l||=\sqrt2$. Mais alors, $\psi_k\neq\psi_l$ puisque la supposition contraire entraînerait $||\varphi_k-\varphi_l||\leq||\varphi_k-\psi_k||+||\varphi_l-\psi_k||<\sqrt2$, une contradiction. On a alors une fonction injective de $B$ dans $D$.
\qed
\end{Pre}

\begthe
Soit $\mathcal H$ un espace de Hilbert séparable. Il existe alors un système complet et orthonormé $\{b_n\}_{n\in I}$ avec $I\subseteq\mathbb N$.
\end{The}

\begpre
Comme $\mathcal H$ est séparable, il existe une famille $D=\{d_n\}_{n\in\mathbb N}$ dense dans $\mathcal H$. On va itérativement construire une famille $\{v_n\}_{n\in J}\subseteq D$ de la manière suivante :
\begin{eqnarray}
\varphi(n+1)=\inf\{k\in\mathbb N:k>\varphi(n)\text{ et }d_k\notin \text{Vect}[v_0,...,v_n]\}\\
\varphi(0)=0, v_{n+1}=d_{\varphi(n+1)}
\end{eqnarray}
Clairement, la famille $\{v_n\}_{n\in\mathbb J}$ est formée de vecteurs linéairement indépendants. De plus, $\text{Vect}[v_n]_{n\in J}$ est dense dans $\mathcal H$, puisque $D\subseteq\text{Vect}[v_n]_{n\in J}$. On forme maintenant une famille de vecteurs orthonormés à partir de $\{v_n\}_{n\in J}$ et $\text{Vect}[v_n]_{n\in J}$.
\beq
b_n=\frac1{||v_n-\sum_{k=0}^{n-1}\langle b_k,v_n\rangle b_k||}\left(v_n-\sum_{k=0}^{n-1}\langle b_k,v_n\rangle b_k\right)
\eeq
Manifestement, $\{b_n\}_{n\in J}$ est une famille de vecteurs orthonormés et $\text{Vect}[b_n]_{n\in J}=\text{Vect}[v_n]_{n\in J}$, de sorte de $\{b_n\}_{n\in J}$ forme un système complet.
\qed\end{Pre}

\begthe
Soit $\{b_n\}_{n\in B}, B\subseteq\mathbb N$ une famille de vecteurs orthonormés d'un espace de Hilbert $\mathcal H$ séparable. Alors, les affirmations suivantes sont équivalentes :
\begin{enumerate}
\item $\{b_n\}_{n\in B}$ est complet
\item $\forall x\in \mathcal H$, $x=\sum_{n\in B}\langle b_n,x\rangle b_n$
\item $\forall x,y\in \mathcal H$, $\langle x,y\rangle=\sum_{n\in B}\langle x,b_n\rangle \langle b_n,y\rangle$
\end{enumerate}
\end{The}

\begpre
Montrer que $1.\implies2.$. Pour $F\subseteq B$, $|F|<\infty$, former les espaces $V_F=\text{Vect}[\{b_n\}_{n\in F}]$. Tous ces espaces sont des sous-espaces vectoriels de $\mathcal H$ fermés, complets et convexes. $\{b_n\}_{n\in B}$ étant complet, $\cup_{F\subseteq B:|F|<\infty}V_F$ est dense dans $\mathcal H$. Soient $x\in \mathcal H$, $\epsilon>0$. Il existe alors un $y\in V_F$ tel que $||x-y||<\epsilon$. A fortiori, notant $x_{V_F}$ la projection de $x$ sur $V_F$ (et de même pour $G$ ensuite), $||x-x_{V_F}||<\epsilon$, et si $F\subseteq G\subseteq B$, $|G|<\infty$, on a :
\beq
||x-x_{V_G}||=\inf\{||x-y||:y\in V_G\}\leq\inf\{||x-y||:y\in V_F\}<\epsilon
\eeq
C'est-à-dire, les projections de $x$ sur les $V_F$ convergent vers $x$. Poser $x_{V_F}=\sum_{n\in F}\lambda_n b_n$. Puisque $x-x_{V_F}\perp V_F$, on doit avoir pour $b_m\in V_F$ :
\beq
0=\langle x,b_m\rangle-\overline\lambda\langle b_m,b_m\rangle
\eeq
d'où $\lambda_m=\langle b_m,x\rangle$ et enfin :
\beq
x=\lim_{n\to\infty}\sum_{k\in B,k\leq n}\langle b_k,x\rangle b_k
\eeq
Montrer que $2.\implies3.$ 
\begin{equation}
\langle x,y\rangle=\sum_{n\in B}\sum_{m\in B}\langle b_n,x\rangle \langle y,b_m\rangle\langle b_n,b_m\rangle=\sum_{n\in B}\langle x,b_n\rangle \langle b_n,y\rangle
\end{equation}
Montrer que $3.\implies1.$. En posant $x=y$, on obtient $||x||=\sum_{n\in B}|\langle x,b_n\rangle|^2<\infty$. Ainsi, la suite $\left(\sum_{k\in B,k\leq n}\right)_{n\in\mathbb N}\langle b_k,x\rangle b_k$ est une suite de Cauchy dont on a clairement la limite $x$ car $\left(\langle x,b_k\rangle\right)_{k\in B}$ est une suite dans $l^2(B)$.
\qed
\end{Pre}
\part{Mesures et intégrales}
\section{Séparation et partition}
Pour $A\subseteq\mathbb R^n$ et $x\in\mathbb R^n$, poser $d:\mathbb R^n\times \mathcal P(\mathbb R^n)\to\mathbb R^+$ :
\beq
(x,A)\mapsto\inf\{|x-y|:y\in A\}
\eeq
\begthe
Soit $A\subseteq\mathbb R^n$. Alors la fonction $d_A:\mathbb R^n\to\mathbb R^+$ :
\beq x\mapsto d(x,A)\eeq
est continue.
\end{The}
\begpre
Soit $A\neq\emptyset$. Soit $\epsilon>0$ et $x,y\in\mathbb R^n$ tels que $|x-y|<\delta$. Supposer (autrement inverser les rôles) que $d_A(y)\leq d_A(x)$. Utilisant l'inégalité triangulaire :
\begin{equation}
|d_A(x)-d_A(y)|=\inf_{z\in A}d(x,z)-\inf_{z\in A}d(y,z)\leq d(x,y)+\inf_{z\in A}d(y,z)-\inf_{z\in A}d(y,z)<\delta
\end{equation}
Prendre $\delta=\epsilon$.
\qed\end{Pre}
\begdef
Un ensemble de $\mathbb R^n$ est dit relativement compact si sa fermeture est compact.
\end{Def}

\begin{Lem}
Soit un compact $K\subset\mathbb R^n$. Il existe alors un ouvert $U$ relativement compact tel que $K\subset U$.\label{rc}
\end{Lem}
\begpre
Si $K=\emptyset$, prendre $U=B_1(x), x\in\mathbb R^n$. Sinon, considérer la famille d'ouverts $\{B_1(x)\}_{x\in K}$ qui recouvre $K$. Extraire une sous-famille finie $F\subset K$ qui recouvre $K$. L'ouvert :
\beq
U=\bigcup_{x\in F}B_1(x)
\eeq
satisfait aux exigences du lemme.
\qed\end{Pre}
\begdef
Soient $K\subset V\subseteq \mathbb R^n$, avec $K$ compact et $V$ ouvert. On dit qu'une fonction $f\in C_c(\mathbb R^n,[0,1])$ sépare $K$ de $\mathbb R^n\backslash V$ et note $K\prec f\prec V$, si $f^{-1}\{1\}$ est un voisinage de $K$ et si $\text{supp}(f)\subset V$.\end{Def}

\begin{Lem}\textbf{Lemme d'Urysohn}
Soient $K\subset V\subseteq\mathbb R^n$, avec $K$ compact et $V$ ouvert. Il existe alors une fonction $f$ telle que $K\prec f\prec V$.\label{ury}
\end{Lem}
\begpre
Par le lemme \ref{rc}, il existe un ouvert $U$ contenant $K$ relativement compact. Remplaçant si nécessaire $V$ par $V\cap U$, on peut supposer que $V$ est relativement compact. La fonction :
\beq g(x)=\frac{d(x,\mathbb R^n\backslash V)}{d(x,\mathbb R^n\backslash V)+d(x,K)}\eeq est manifestement définie pour tout $x\in \mathbb R^n$ et continue comme combinaison de fonctions continues. De plus, $g|_K=1$ et $g|_{\mathbb R^n\backslash V}=0$. Soient alors les ouverts $W=g^{-1}]2/3,1]$ et $U=g^{-1}]1/3,1]$. Clairement, $K\subset W\subset \overline{U}\subset V$ et la fonction :
\beq
f(x)=\frac{d(x,\mathbb R^n\backslash U)}{d(x,\mathbb R^n\backslash U)+d(x,W)}
\eeq
satisfait aux critères du lemme.
\qed
\end{Pre}

\begdef
Soit $K$ un compact de $\mathbb R^n$ et $\{V_n\}_{1\leq n\leq m}$ une collection finie d'ensembles ouverts qui recouvrent $K$. Une famille de $m$ fonctions $f_n\prec V_n$ telles que :
\beq
\sum_{n=1}^mf_n(x)=1, \forall x\in K
\eeq
est appelée une partition de $K$ subordonnée au recouvrement $\{V_n\}_{1\leq n\leq m}$.
\end{Def}

\begin{Cor}
Soit $K\subset \mathbb R^n$ compact et $\{V_n\}_{1\leq n\leq m}$ une collection finie d'ensembles ouverts qui recouvrent $K$. Il existe alors une partition de $K$ subordonnée à $\{V_n\}_{1\leq n\leq m}$.\label{partition}
\end{Cor}
\begpre
Soit $x\in K$. Il existe $V_{n_x}$ du recouvrement tel que $x\in V_n$. Par le lemme d'Urysohn \ref{ury}, il existe une fonction $g_x$ telle que $\{x\}\prec g_x\prec V_{n_x}$. L'ensemble $K_x=g_x^{-1}\{1\}$ est alors un voisinage compact de $\{x\}$. Comme $K$ est compact et puisque $\{\accentset{\circ}{K_x}\}_{x\in K}$ recouvre $K$, il existe une sous-collection finie $\{K_{x_j}\}_{j=1,...,p}$ qui recouvre $K$. Pour chaque $V_n$ du recouvrement initiale, poser :
\beq C_n=\bigcup_{K_{x_j}\subset V_n, 1\leq i\leq p}K_{x_j}\eeq Tous les $C_n$ sont compacts et leur collection recouvre $K$. De plus, $C_n\subset V_n$, $n=1,...,m$. Une nouvelle application du lemme d'Urysohn livre alors $m$ fonctions $h_n$ telles que $C_n\prec h_n\prec V_n$. Poser alors $f_1=h_1$ et $f_n=h_n\prod_{k=1}^{n-1}(1-h_k)$, pour $n\geq 2$. Clairement, $f_n\prec V_n$ pour $n=1,...,m$ et :
\beq
\sum_{n=1}^mf_n=1-\prod_{n=1}^m(1-h_n)
\eeq
De plus, si $x\in K$, $x\in C_n$ pour au moins un $n$, de sorte que $h_n(x)=1$, c'est-à-dire la propriété espérée.
\qed\end{Pre}

\section{Mesures et fonctionnelles positives}
\begdef
Une fonctionnelle $\Phi:C_c(\mathbb R^n,\mathbb R)\to\mathbb R$ est dite positive si $f\geq0$ implique $\Phi(f)\geq0$.
\end{Def}

\begdef Soit $\Phi:C_c(\mathbb R^n,\mathbb R)$ une fonctionnelle positive. Soient $K,V\subset\mathbb R^n$ avec $K$ compact, $V$ ouvert. Définir :
\begin{eqnarray}
\mu(K)=\inf\{\Phi(f):K\prec f\}\in\mathbb R^+\\
\mu(V)=\sup\{\Phi(f):f\prec V\}\in\mathbb R^+\cup\{\infty\}
\end{eqnarray}
\end{Def}

\begin{Lem}
Soient $K,V\subset\mathbb R^n$, $K$ compact, $V$ ouvert. Alors :
\begin{eqnarray}
\mu(K)=\inf\{\mu(W):K\subset W, W\text{ ouvert}\}\\
\mu(V)=\sup\{\mu(C):C\subset V, C\text{ compact}\}
\end{eqnarray}
\end{Lem}
\begpre
Si $K\subset W\subset\mathbb R^n$, $K$ compact, $W$ ouvert, alors par le lemme d'Urysohn \ref{ury}, il existe une fonction $K\prec f\prec W$. Puisque $f^{-1}\{1\}$ est un voisinage de $K$, on a :
\beq
\mu(K)=\inf\{\Phi(f):K\prec f\}\geq\inf\{\mu(U):K\subset U, U\text{ ouvert}\}\geq\mu(K)
\eeq Similairement, puisque pour une fonction $f\prec V$, on a $\text{supp}(f)\subset V$ et que $\text{supp}(f)$ est compact, on :
\beq
\mu(V)=\sup\{\Phi(f):f\prec V\}\leq\sup\{\mu(C):C\subset V, C\text{ compact}\}\leq\mu(V)
\eeq
\qed\end{Pre}

\begdef On définit une mesure intérieure $\mu_*:\mathcal P(\mathbb R^n)\to\mathbb R_+\cup\{\infty\}$ et une mesure extérieure $\mu^*:\mathcal P(\mathbb R^n)\to\mathbb R_+\cup\{\infty\}$ par :
\beq
\mu_*(E)=\sup\{\mu(K):K\subseteq E,K\text{ compact}\} \text{ et } \mu^*(E)=\inf\{\mu(V):E\subseteq V,V\text{ ouvert}\}
\eeq
\end{Def}

\begin{Lem}
Si $\{F_n\}_{n\in\mathbb N}$ est une suite de sous-ensembles de nombres réels positifs, alors :
\beq
\inf\left\{\sum_{n=0}^\infty a_n|\forall k : a_k\in F_k\right\}=\sum_{n=0}^\infty\inf F_n\in\mathbb R_+\cup\{\infty\}
\eeq\label{eqcheloue}
\end{Lem}
\begpre
Noter $A$ le terme de gauche, $B$ le terme de droite. D'abord, remarquer que $B$ minore l'ensemble $\left\{\sum_{n=0}^\infty a_n|\forall k : a_k\in F_k\right\}$ donc $B\leq A$. Si $B=\infty$, alors $A=B$. Supposer donc $B<\infty$. Alors, $\forall n\geq0$, $\inf F_n<\infty$. Soit $x>B$ et poser $\epsilon=x-B$.
\beq
\forall n\in\mathbb N, \exists a_n\in F_n:a_n<\inf F_n+\epsilon/2^{n+1}
\eeq
Alors, $A\leq\sum_{n\geq0}a_n\leq B+\epsilon=x$. Puisque $x$ est arbitraire, $A\leq B$.
\qed\end{Pre}

\begin{Lem}
La mesure intérieure $\mu_*$ est sur-additive alors que la mesure extérieure est sous-additive. C'est-à-dire que si $\{E_n\}_{n\in\mathbb N}$ est une suite de sous-ensembles de $\mathbb R^n$ deux à deux disjoints, on a :
\beq
\mu_*\left(\bigcup_{n=0}^\infty E_n\right)\geq\sum_{n=0}^\infty \mu_*(E_n)
\eeq
et
\beq
\mu^*\left(\bigcup_{n=0}^\infty E_n\right)\leq\sum_{n=0}^\infty \mu^*(E_n)
\eeq\label{sursous}
\end{Lem}

\begpre
Pour montrer la première inégalité, il suffit de montrer que pour tout $m\in\mathbb N$ :
\beq
\mu_*\left(\bigcup_{n=0}^m E_n\right)\geq\sum_{n=0}^m \mu_*(E_n)
\eeq
Par définition de la mesure extérieure,
\beq
\sum_{n=0}^m\mu_*(E_n)=\sup\left\{\sum_{n=0}^m\mu(K_n):K_n\subseteq E_n, K_n\text{ compact}\right\}
\eeq
Puisque l'union finie de compact est contenue dans l'union finie des $(E_n)_{n=0}^m$, il suffit de montrer que :
\beq
\mu\left(\bigcup_{n=0}^mK_n\right)\geq\sum_{n=0}^m\mu(K_n)
\eeq
Plus simplement, il suffit de montrer que pour deux compacts disjoints $K_1$ et $K_2$, $\mu(K_1\cup K_2)\geq \mu(K_1)+ \mu(K_2)$. Soit alors $K_1\cup K_2\prec f$. Comme $K_1$ est disjoint de $K_2$, le lemme d'Urysohn \ref{ury} assure qu'il existe $f_1$ telle que $K_1\prec f_1\prec\mathbb R^N\backslash K_2$. Par le même argument, il existe $f_2$ telle que $K_2\prec f_2\prec \mathbb R^N\backslash\text{supp}(f_1)$. Alors $K_1\cup K_2\prec f(f_1+f_2)\leq f$, $K_1\prec ff_1$ et $K_2\prec ff_2$, de sorte que :
\beq
\Phi(f)\geq\Phi(ff_1)+\Phi(ff_2)\geq\mu(K_1)+\mu(K_2)
\eeq
En prenant l'infimum sur toutes les fonctions $f$, on trouve $\mu(K_1\cup K_2)\geq \mu(K_1)+ \mu(K_2)$.

Soit maintenant une suite d'ouverts $\{V_n\}_{n\in\mathbb N}$, $E_n\subseteq V_n$. Clairement, $\bigcup_{n=0}^\infty E_n\subseteq\bigcup_{n=0}^\infty V_n$. Soit $f\prec \bigcup_{n=0}^\infty V_n$. Puisque $f$ est à support compact, le corollaire \ref{partition} implique qu'il existe une partition $(f_n)_{n=0}^m$ de $\text{supp}(f)$ subordonnée au recouvrement fini $\{V_n\}_{n=0}^m$. Alors :
\beq
\Phi(f)=\Phi\left(f\sum_{n=0}^mf_n\right)=\sum_{n=0}^m\Phi(ff_n)\leq\sum_{n=0}^m\mu(V_n)\leq\sum_{n=0}^\infty \mu(V_n)
\eeq
Il suit en prenant l'infimum à gauche :
\beq
\mu\left(\bigcup_{n=0}^\infty V_n\right)\leq\sum_{n=0}^\infty\mu(V_n)\eeq
Par le lemme \ref{eqcheloue} :
\beq
\mu^*\left(\bigcup_{n=0}^mE_n\right)\leq\inf\left\{\mu\left(\bigcup_{n=0}^mV_n\right):\forall k, E_k\subseteq V_k\right\}\leq\inf\left\{\sum_{n=0}^\infty \mu(V_n):\forall k, E_k\subseteq V_k\right\}=\sum_{n=0}^\infty\mu^*(E_n)
\eeq
\qed\end{Pre}

\begdef
Soit $\Phi: C_c(\mathbb R^N,\mathbb R)\to\mathbb R$ une fonctionnelle positive ainsi que ses mesures intérieures et extérieures $\mu_*$ et $\mu^*$. Un ensemble $E\subseteq\mathbb R^N$ est dit mesurable si et seulement si pour tout compact $K\subset \mathbb R^N$ :
\beq
\mu_*(K\cap E)=\mu^*(K\cap E)
\eeq
La collection des ensembles mesurables est dénotée $\Sigma$.
\end{Def}

\begin{Lem}
Soit $\{E_n\}_{n\in\mathbb N}\subseteq\Sigma$ une suite d'ensembles mesurables deux à deux disjoints. Alors $\bigcup_{n=0}^\infty E_n\in\Sigma$ et :
\beq
\mu_*\left(\bigcup_{n=0}^\infty E_n\right)=\sum_{n=0}^\infty\mu_*(E_n)
\eeq \label{addisub}
\end{Lem}

\begpre
Commencer par montrer que l'union des $\{E_n\}$ est dans $\Sigma$. Prendre $K\subset\mathbb R^n$ compact. Alors :
\begin{equation}\begin{split}
\mu_*\left(K\cap\bigcup_{n\in\mathbb N}E_n\right)&=\mu_*\left(\bigcup_{n\in\mathbb N}K\cap E_n\right)
\\& \leq\mu^*\left(\bigcup_{n\in\mathbb N}K\cap E_n\right)\text{, car $\mu_*\leq\mu^*$}
\\& \leq \sum_{n\in\mathbb N}\mu^*(K\cap E_n)\text{, par sous-additivité \ref{sursous}}
\\& =\sum_{n\in\mathbb N}\mu_*(K\cap E_n)\text{, car $E_n\in\Sigma$}
\\& \leq \sum_{n\in\mathbb N}\mu_*(E_n)\text{, par monotonie}
\\& \leq \mu_*\left(\bigcup_{n\in\mathbb N}E_n\right)\text{, par sur-additivité \ref{sursous}}
\end{split}\end{equation}
Ainsi, d'une part : 
\beq
\mu_*\left(K\cap\bigcup_{n\in\mathbb N}E_n\right)\leq\mu^*\left(K\cap\bigcup_{n\in\mathbb N}E_n\right)\leq\sum_{n\in\mathbb N}\mu_*(K\cap E_n)\leq\mu_*\left(K\cap\bigcup_{n\in\mathbb N}E_n\right)
\eeq
donc $\bigcup_{n\in\mathbb N}E_n\in\Sigma$. D'autre part, en prenant le supremum sur tous les compacts $K\subseteq\bigcup_{n\in\mathbb N}E_n$ :
\beq
\mu_*\left(\bigcup_{n\in\mathbb N}E_n\right)\leq \sum_{n\in\mathbb N}\mu_*(E_n)\leq\mu_*\left(\bigcup_{n\in\mathbb N}E_n\right)
\eeq
\qed\end{Pre}

\begin{Lem} Soient $E,F\in\Sigma$, alors $E\backslash F\in \Sigma$.\label{comple}
\end{Lem}
\begpre
Soit $K\subset\mathbb R^N$ compact. Il reste à montrer que $\mu_*(K\cap(E\backslash F))\geq\mu^*(K\cap(E\backslash F))$. Comme $K\cap(E\backslash F)\subseteq K$, $K\cap E\subseteq K$ et $K\cap F\subseteq K$, il existe des ouverts $V_E$ et $V_F$ de mesures extérieures finies tels que $K\cap E\subseteq V_E$ et $K\cap F\subseteq V_F$. Soient des compacts $K\subseteq K\cap E$ et $K_F\subseteq K\cap F$. On a alors :
\beq
K_E\backslash V_F\subseteq K\cap E\backslash K\cap F=K\cap(E\backslash F)\subseteq V_E\backslash K_F
\eeq
Remarquer alors que $K_E\backslash V_F$ est compact, $V_E\backslash K_F$ est ouvert et que :
\beq
V_E\backslash K_F=(V_E\backslash K_E)\cup (K_E\backslash V_F) \cup (V_F\backslash K_F)
\eeq
Par monotonie et sous-additivité de la mesure extérieure, on a :
\begin{equation}\begin{split}
\mu^*(K\cap(E\backslash F))&\leq\mu^*(V_E\backslash K_F)\\ &\leq \mu^*(V_E\backslash K_E)+ \mu^*(K_E\backslash V_F) + \mu^*(V_F\backslash K_F) \\ &\leq \mu^*(V_E)-\mu^*(K_E)+\mu_*(K\cap(E\backslash F))+\mu^*(V_F)-\mu^*(K_F)
\end{split}\end{equation}
en utilisant la mesurabilité des compacts. Avec la mesurabilité de $K\cap E$ et $K\cap F$, on obtient de plus :
\beq
\inf\{\mu^*(V_E)-\mu^*(K_E)+\mu^*(V_F)-\mu^*(K_F):K_E\subseteq K\cap E\subseteq V_E, K_F\subseteq K\cap F\subseteq V_F\}=0
\eeq
\qed\end{Pre}

\begin{The}
Soit $\Phi:C_c(\mathbb R^N,\mathbb R))\to\mathbb R$ une fonctionnelle positive, sa mesure intérieure $\mu_*$ et extérieure $\mu^*$ ainsi sur les ensembles mesurables $\Sigma$. Alors $\Sigma$ est une $\sigma$-algèbre borelienne. De plus $\mu$ définit alors une mesure régulière et complète sur $\Sigma$.
\end{The}

\begpre
Clairement, $\emptyset\in\Sigma$. Le complémentaire d'une ensemble mesurable est mesurable par le lemme \ref{comple}. Soit maintenant $\{E_n\}_{n\in\mathbb N}\subset\Sigma$. Poser $F_0=E_0$ et pour $n\geq1$, $F_n=E_n\backslash(\bigcup_{0\geq m\geq n-1}F_m)$. $\{F_n\}_{n\in\mathbb N}\subset \Sigma$ par le lemme \ref{addisub} et $\bigcup_{n\in\mathbb N}E_n=\bigcup_{n\in\mathbb N}E_n$. Ainsi, $\Sigma$ est une $\sigma$-algèbre. Ensuite, puisque les fermés sont mesurables, les ouverts le sont aussi par le lemme \ref{comple}, c'est-à-dire $\Sigma$ est borelienne. Montrer maintenant que $\mu$ définit bien une mesure sur $\Sigma$. Soit $E\in\Sigma$. Par sous-additivité de la mesure intérieure, la mesurabilité de E et des $B_n(0)$ ainsi que la sur-additivité de $\mu_*$, on a :
\begin{equation}\begin{split}
\mu^*(E)&=\mu^*\left(\left(\bigcup_{n\in\mathbb N^*}B_n(0)\right) \cap E\right)\\
&=\mu^*\left(\left(\bigcup_{n\in\mathbb N^*}B_n(0)\backslash B_{n-1}(0) \right) \cap E\right)
\\& \leq\sum_{n\in\mathbb N^*}\mu^*(B_n(0)\backslash B_{n-1}(0) \cap E)
\\& =\sum_{n\in\mathbb N^*}\mu^*((B_n(0)\cap(B_n(0)\backslash B_{n-1}(0) \cap E))
\\& = \sum_{n\in\mathbb N^*}\mu_*((B_n(0)\cap(B_n(0)\backslash B_{n-1}(0) \cap E))
\\& = \sum_{n\in\mathbb N^*}\mu_*(B_n(0)\backslash B_{n-1}(0) \cap E)
\\&=\mu_*\left(\left(\bigcup_{n\in\mathbb N^*}B_n(0)\backslash B_{n-1}(0) \right) \cap E\right)
\\& = \mu_*(E)
\end{split}\end{equation}
En utilisant le lemme \ref{addisub}, il est clair que $\mu$ définit bien une mesure. La régularité de $\mu$ est claire par les dernières équations, et la complétude découle de la définition de $\mu$.
\qed\end{Pre}

\begdef
Si $\Phi$ est l'intégrale de Riemann, alors $\mu$ est la mesure de Lebesgue.
\end{Def}

\begin{The}
\textbf{Théorème de Riesz-Kakutani} Soit $\Phi:C_c(\mathbb R^n)\to\mathbb R$ une fonctionnelle positive. Il existe alors une mesure $\mu$ régulière et complète, définie sur une $\sigma$-algèbre $\Sigma\subseteq\mathcal P(\mathbb R^n)$ borelienne, telle que :
\beq
\forall f\in C_c(\mathbb R^n) : \Phi(f)=\int_{\mathbb R^n}f\mathrm{d}\mu\eeq\label{rk}
\end{The}

\begpre
Il est suffisant de prouver le théorème dans le cas où $f$ prend des valeurs réelles. En fait, il suffit de prouver que :
\beq
\forall f\in C_c(\mathbb R^N,\mathbb R) : \Phi f\leq \int_{\mathbb R^N}f\mathrm{d}\mu
\eeq
En effet, dès lors :
\beq
-\Phi f=\Phi(-f)\leq\int_{\mathbb R^N}(-f)\mathrm{d}\mu=-\int_{\mathbb R^N}f\mathrm{d}\mu
\eeq
Soit $K=\text{supp}(f)$ et $[a,b]$ son image. Soit $\epsilon>0$ et pour chaque $i=0,...,n$ avec $y_i-y_{i-1}<\epsilon$ :
\beq
y_0<a<y_1<...<y_n=b
\eeq
Définir pour chaque $i=1,..,n$ :
\beq
E_i=\{x\in\mathbb R^N:y_{i-1}<f(x)<y_i\}\cap K
\eeq
Car $f$ est continue, $f$ est Borel-mesurable et les ensembles $E_i$ sont des ensembles de Borel disjoints d'union $K$. Aussi, il existe des ouverts $V_i$ tels que $E_i\subseteq V_i$ et $\mu(V_i)<\mu(E_i)+\epsilon/n$. De plus :
\beq \forall x\in V_i: f(x)<y_i+\epsilon\eeq
Par le corollaire \ref{partition}, il existe, pour chaque $i$, $h_i\prec V_i$ une partition de $K$ extraite de la famille finie des $\{V_i\}_{i=1}^n$. En particulier, $f=\sum h_i f$ d'où :
\beq
\mu(K)\leq\Phi\left(\sum_{i=1}^n h_i\right)=\sum_{i=1}^n \Phi h_i
\eeq
D'autre part, puisque $h_if\leq(y_i+\epsilon)h_i$ et $y_i<f(x)+\epsilon$ sur $E_i$ :
\begin{equation}\begin{split}
\Phi f&=\sum_{i=1}^n\Phi(h_if)
\leq\sum_{i=1}^n(y_i+\epsilon)\Phi h_i
= \sum_{i=1}^n(|a|+y_i+\epsilon)\Phi h_i-|a|\sum_{i=1}^n\Phi h_i
\leq \sum_{i=1}^n(|a|+y_i+\epsilon)(\mu(E_i)+\epsilon/n)-|a|\mu(K)
\\ &= \sum_{i=1}^n(y_i-\epsilon)\mu(E_i)+2\epsilon\mu(K)+\frac{\epsilon}n\sum_{i=1}^n(|a|+y_i+\epsilon)
\leq \int_{\mathbb R^N}f\mathrm{d}\mu+\epsilon(2\mu(K)+|a|+b+\epsilon)
\end{split}\end{equation}
\qed\end{Pre}

\section{Résultats de densité}

\begdef
Une mesure $\mu$ définie sur une $\sigma$-algèbre $\Sigma\subseteq\mathcal P(\mathbb R^n)$ borélienne est dite intérieurement régulière si :
\beq
\forall E\in\Sigma, \mu(E)=\sup\{\mu(K):K \text{ compact et } K\subset E\}
\eeq
Elle est dite extérieurement-régulière si
\beq
\forall E\in\Sigma, \mu(E)=\inf\{\mu(V):V \text{ ouvert et } E\subset V\}
\eeq
Elle est enfin régulière si elle est simultanément extérieurement et intérieurement régulière, et localement finie si :
\beq
\forall x\in\mathbb R^n,\exists \text{ un ouvert }U\in\Sigma : x\in U\text{ et } \mu(U)<\infty
\eeq
\end{Def}

\begin{Lem}
Une mesure régulière et localement finie $\mu$ dans $\mathbb R^n$ assigne à tout compact $K$ de $\mathbb R^n$ une mesure finie.
\end{Lem}
\begin{Pre}
Pour chaque $x\in K$ il existe un ouvert $V_x$ contenant $x$ et de mesure finie. L'ensemble des tels $V_x$ recouvre $K$, on peut donc en extraire un sous-recouvrement fini $(V_i)_{i=1,..p}$, associé aux $(x_i)_{i=1,..p}$. Poser :
\beq
E_i=K\bigcap_{i=1}^p\left(V_i\backslash \bigcup_{j=1}^{i-1} V_j\right)
\eeq
La régularité de $\mu$ implique la mesurabilité de ces ensembles. De plus, $K=\bigcup_{i=1}^pE_i$ et cette union est disjointe. Ainsi :
\beq
\mu(K)=\sum_{i=1}^p\mu(E_i)\leq\sum_{i=1}^p\mu(V_i)<\infty
\eeq
\qed\end{Pre}

\begin{Lem}
L'espace des fonctions simples Lebesgue intégrables est dense dans $L^p(\mathbb R^n,\mu)$, pour une mesure régulière et localement finie $\mu$.
\label{densite1}\end{Lem}

\begin{Pre}
Trouvons $\varphi\in C_c(\mathbb R^n,\mathbb R^+)$ pour une fonction $f\in L^p(\mathbb R^n,\mu)$ positive. Poser pour $m\in\mathbb N$ :
\beq
f_m:x\mapsto \chi_{B_m(0)}(x)\min\{f(x),m\}
\eeq
Alors, pour chaque $m\geq0$, $f_m\in L^1(\mathbb R^n,\mu)$. Puisque $|f-f_m|^p\leq|f|^p$, le théorème de la convergence dominée implique que $(f_m)$ converge vers $f$ dans la norme $||\cdot||_p$. Il existe donc $k\in\mathbb N$ tel que $m\geq k$ implique $||f_k-f||_p<\epsilon/2$. Par définition de l'intégrale de Lebesgue, et puisque $f_m\in L^1(\mathbb R^n,\mu)$, on peut approcher $f_m$ par une fonction étagée simple positive, celle-ci se laissant approcher par une fonction $\varphi\in C_c(\mathbb R^n,\mathbb R_+)$, telle que $||f_m-\varphi||_1<\frac{\epsilon^p}{2^pm^{p-1}}$. Puisque $f_m$ est majorée par $m$, on peut le supposer aussi pour $\varphi$. On a alors $|f_m(x)-\varphi(x)|\leq m$, et $|f_m-\varphi|^p\leq m^{p-1}|f_m-\varphi|$. Par conséquent,
\beq
||f_m-\varphi||_p^p\leq m^{p-1}||f_m-\varphi||_1\leq\left(\frac\epsilon2\right)^p
\eeq
C'est-à-dire $||f-\varphi||_p<\epsilon$.
\qed\end{Pre}

\begin{Lem}
L'espace des combinaisons linéaires de fonctions indicatrices sur des ouverts de mesures finies ou sur des compacts est dense dans $L^p(\mathbb R^n,\mu)$, pour une mesure régulière et localement finie $\mu$.\label{densite2}
\end{Lem}

\begin{Pre}
Soient $\epsilon>0$ et $E$ un ensemble mesurable de $\mathbb R^n$ et de mesure finie. Puisque $\mu$ est régulière, il existe un compact $K_E$ et un ouvert $V_E$ tel que $K_E\subseteq E\subseteq V_E$ et :$$\mu(V_E)-\epsilon^p\leq\mu(E)\leq\mu(K_E)+\epsilon^p$$. Il suit :
\beq
||\chi_{V_E}-\chi_{E}||_p,||\chi_{K_E}-\chi_{E}||_p<\epsilon
\eeq
Utiliser le lemme \ref{densite1} pour conclure.
\qed\end{Pre}

\begin{Lem}
Pour une mesure régulière et localement finie $\mu$, l'espace des fonctions continues à support compact est dense dans $L^p(\mathbb R^n,\mu)$ :
\beq \overline{C_c(\mathbb R^n)}^{||\cdot||_{L^p(\mathbb R^n,\mu)}}=L^p(\mathbb R^n,\mu)\eeq\label{densite3}
\end{Lem}

\begin{Pre}
Grâce au lemme \ref{densite2}, on peut se concentrer sur les indicatrices d'ouverts de mesure finie ou de compact. Par le lemme d'Urysohn \ref{ury}, il existe $f\in C_c(\mathbb R^n)$ tel que $K\prec f\prec V$. Par monotonie de l'intégrale de Lebesgue, on a dans le cas où $\mu^K+\epsilon^p\geq\mu(V)$ :
\beq
||\chi_K-f||_1, ||\chi_V-f||_1\leq \epsilon^p
\eeq
ce qui implique :
\beq||\chi_K-f||_p, ||\chi_V-f||_1\leq \epsilon\eeq
\qed\end{Pre}

\begin{The} 
Pour une mesure régulière et localement finie $\mu$, l'espace des fonctions continues à support compact et infiniment continument dérivable est dense dans $L^p(\mathbb R^n,\mu)$ :
\beq \overline{C_c^\infty(\mathbb R^n)}^{||\cdot||_{L^p(\mathbb R^n,\mu)}}=L^p(\mathbb R^n,\mu)\eeq
\label{densite_smooth}\end{The}

\begin{Pre}
Partir du lemme \ref{densite3}. Un compact dans $\mathbb R^n$ étant fermé et borné, il doit pour une fonction $f\in C_c(\mathbb R^n)$ y avoir $L>0$ tel que $\text{supp}(f)\subseteq[-L,L]^n$. La fonction étant uniformément continue sur ce dernier ensemble, il doit exister une fonction simple du type :
\beq
s=\sum_{l=1}^ms_l\chi_{d_n+[0,p[^n}
\eeq
telle que :
\beq
|f-s|<\frac{\epsilon}{\mu\left([-L,L]^n\right)^{1/p}}\implies||f-s||_p<p
\eeq
Finalement, pour un intervalle $[a,b[^n$, considérons les fonctions lisses du type $(f_n)=\left(\prod_{k=1}^mf_{m,k})\right)$ où $ m\in\mathbb N^*$ et 
\beq
f_{m,k}(x_k)=\chi_{]a-1/m,b[}(x_k)e^\frac1{m^2\left(x_k+1/m-a\right)\left(x_k-b\right)}
\eeq
Alors, $\lim_{m\to\infty}f_m(x)=\chi_{[a,b[^n}(x)$ simplement et chaque élément de la suite est intégrable. Par le théorème de la convergence dominée, la suite tend dans $L^p(\mathbb R^n)$ vers $\chi_{[a,b]^n}$.
\qed\end{Pre}

\begin{Cor}
Si $\lambda$ est la mesure de Lebesgue, alors l'espace de Schwartz est dans $L^p(\mathbb R^n)$ :
\beq \overline{\mathscr S(\mathbb R^n)}^{||\cdot||_{L^p(\mathbb R^n,\lambda)}}=L^p(\mathbb R^n,\lambda)\eeq
\end{Cor}




\end{document}

