\documentclass[a4paper,10pt,preprint,aps,tightenlines,showpacs,superscriptaddress]{revtex4}	%Rajouter a4paper,12pt suites à relecture des instructions
\usepackage[top=2cm, bottom=2cm, left=1cm, right=1cm]{geometry}	% new package
\usepackage[dvips]{graphicx}
\usepackage{dcolumn}
\usepackage{epsfig}
\usepackage{color}
\usepackage{bm}
\usepackage{gensymb}
\usepackage{amsmath}
\usepackage{amsthm}
\usepackage{appendix}
\usepackage[english]{babel}
\usepackage[T1]{fontenc}
\usepackage{fancyhdr}
\usepackage{amsfonts}
\usepackage{comment}
\usepackage{accents}
\usepackage{mathrsfs}  

\def\bra#1{\mathinner{\langle{#1}|}}
\def\ket#1{\mathinner{|{#1}\rangle}}
\def\braket#1{\mathinner{\langle{#1}\rangle}}

\graphicspath{{tpillustrations/}}

\def\slash#1{\not\!#1}
\def\slashb#1{\not\!\!#1}
\def\delsla{\not\!\partial}

\newcommand{\red}[1]{\textcolor{red}{#1}}
\newcommand{\blue}[1]{\textcolor{blue}{#1}}
\newcommand{\green}[1]{\textcolor{green}{#1}}
\newcommand{\magenta}[1]{\textcolor{magenta}{#1}}

\newcommand{\sla}[1]{\not\! #1}
\def\ohalf{{\textstyle{1\over 2}}}
\def\half{{\textstyle{1\over 2}}}
\def\vqhalf{{\textstyle{\vec{Q}\over 2}}}
\def\qhalf{{\textstyle{Q\over 2}}}
\def\osix{{\textstyle{1\over 6}}}
\def\vqsix{{\textstyle{\vec{Q}\over 6}}}
\def\thalf{{\textstyle{3\over 2}}}
\def\fourth{{\textstyle{1\over 4}}}
\def\tfor{{\textstyle{3\over 4}}}

\newcommand{\inp}[2]{\langle{#1}|{#2}\rangle}
\newcommand{\vev}[1]{\langle{#1}\rangle}
\newcommand{\slas}[1]{\not\!{#1}}
\def\ohalf{{\textstyle{1\over 2}}}
\def\thalf{{\textstyle{3\over 2}}}
\def\fhalf{{\textstyle{5\over 2}}}
\def\shalf{{\textstyle{7\over 2}}}
\def\nhalf{{\textstyle{9\over 2}}}


\newcommand\Tr{\,{\rm Tr}\,}
\newcommand\re{\Re\mbox{e}}
\newcommand\im{\Im\mbox{m}}
\newcommand{\beq}{\begin{equation}}
\newcommand{\eeq}{\end{equation}}
\newcommand{\Hcal}{\mathcal H}

\pagestyle{fancy}
\lhead{\scriptsize{ÉCOLE POLYTECHNIQUE FÉDÉRALE DE LAUSANNE\\
3ème année de Physique\\
Baptiste CLAUDON}}
\rhead{\empty}
\renewcommand{\headrulewidth}{0pt}

\theoremstyle{definition}
\newtheorem{Def}{Définition}
\newcommand{\begdef}{\begin{Def}}
%\newcommand{\enddef}{\end{Def}}
\newtheorem{Prop}{Proposition}
\newcommand{\begpro}{\begin{Prop}}
%\newcommand{\endpro}{\end{Prop}}
\newtheorem{The}{Théorème}
\newcommand{\begthe}{\begin{The}}
%\newcommand{\endthe}{\end{The}}
\newtheorem*{Pre}{Preuve}
\newcommand{\begpre}{\begin{Pre}}
\newtheorem{Ex}{Exemple}
\newtheorem{Post}{Postulat}
\newtheorem{Cor}{Corollaire}
\newcommand{\begcor}{\begin{Cor}}
\newtheorem{Lem}{Lemme}

\begin{document}

\begin{titlepage}
   \vspace*{\stretch{1.0}}
   \begin{center}
      \Huge\textbf{Analyse Fonctionnelle\\pour Physiciens}\\
      %\Large{Introduction aux Ondes\\ et à la Mécanique Quantique}\\ 
      \text{ }\newline
      \large{Baptiste Claudon}\\
      \today\\
      \vspace{5cm}Notes personnelles basées sur le cours de Simon Bossoney.\\Suit les cours d'Analyse I à III de Joachim Stubbe et le cours d'Analyse IV de Matthias Ruf.
   \end{center}
   \vspace*{\stretch{2.0}}
\end{titlepage}

\tableofcontents\newpage
\newpage
\part{Espaces Fonctionnels}
\section{Le théorème de Stone-Weierstrass}
\begthe\textbf{Théorème de Diniz}
Soit $(f_n)$ une suite de fonctions réelles et continues définies sur un compact $K\subset\mathbb R^n$ et convergent simplement et de manière monotone vers $f\in C(K,\mathbb R)$. Alors cette suite converge uniformément vers $f$.\label{diniz}
\end{The}
\begpre
Choisir, sans perte de généralité que $(f_n)$ est décroissante et converge simplement vers $0$. Soit $\epsilon>0$. Poser pour $n\in\mathbb N$ : 
\beq
V_n=\{x\in K:f_n(x)<\epsilon\}
\eeq
Par continuité des fonctions de la suite, tous ces ensembles sont des ouverts. Puisque la suite tend vers $0$, on a : 
\beq
K=\bigcup_{n\in\mathbb N}V_n
\eeq
$K$ étant compact, il existe un nombre $F\in\mathbb N$ tel que
\beq
K=\bigcup_{n=0}^FV_n=K
\eeq
Puisque la suite est monotone décroissante, on a que $m<n$ implique $V_m\subseteq V_n$, donc $V_F=K$.
\qed
\end{Pre}

\begdef
Soit $F$ une famille de fonctions définies sur un ensemble $X\subset\mathbb R^n$. On dit que $F$ sépare $X$ si :
\beq
\forall x,y\in X, x\neq y,\exists f\in F:f(x)\neq f(y)
\eeq
\end{Def}
\begdef
On dit que $F$ ne s'annule pas sur $X$ si :
\beq
\forall x\in X\exists f\in F:f(x)\neq0
\eeq
\end{Def}
\begdef
Si $B$ est un sous-ensemble d'une $\mathbb K$-algèbre $A$, alors la $\mathbb K$-algèbre engendrée par $B$, $\mathcal A_\mathbb K(B)$ est la plus petite $\mathbb K$-algèbre contenant $B$.
\end{Def}
\begthe\textbf{Théorème de Stone-Weierstrass} Soit $X\subset\mathbb R^n$ un ensemble compact et soit $F\subseteq C(X,\mathbb R)$ une famille de fonctions qui sépare $X$ et qui ne s'annule pas sur $X$. Alors l'algèbre réelle $\mathcal A_\mathbb K(F)$ engendrée par $F$ est uniformément dense dans $C(X,\mathbb R)$ : 
\beq
\overline{\mathcal A_\mathbb K(F)}^{||\cdot||_\infty}=C(X,\mathbb R)
\eeq
\label{sw}
\end{The}

\begpre 
%%	ÉTAPE 1
\begin{Lem}
Il existe une suite $(P_n)$ de polynômes sur $[-1,1]$ qui converge uniformément vers $x$.\label{step1}
\end{Lem}
Définir la suite de polynômes par $P_0=0$ et :
\begin{eqnarray}
\forall n\geq0:P_{n+1}(x)=P_n(x)+\frac12(x^2-P_n(x)^2)
\end{eqnarray}
On déduit alors par récurrence que $\forall n\in\mathbb N,\forall x\in[-1,1] :0\leq P_n(x)\leq|x|$. Alors, on déduit que la suite $(P_n)$ est croissante. Puisque qu'elle est majorée elle doit converger. La limite doit être point fixe de l'application donc est $x\mapsto|x|$. Par le théorème de Diniz \ref{diniz}, la convergence est uniforme.
%%	ÉTAPE 2
\begin{Lem}
Si $f,g_i\in\overline{\mathcal A(F)}$, $i=1,...,n$, alors $|f|, \min_{i=1,...,n}g_i,\max_{i=1,...,n}g_i$ appartiennent aussi à $\overline{\mathcal A(F)}$. \label{step2}
\end{Lem}
Si $f\neq0$, poser $h=\frac{f}{||f||}$. Noter $h\in\overline{\mathcal A(F)}$ et a son image dans $[-1,1]$. Par le lemme \ref{step1}, il existe un polynôme $P$ tel que $P(h)$ converge uniformément vers $\frac{|f|}{||f||}$. Pour $g_1,g_2\in\overline{\mathcal A(F)}$, utiliser :
\beq
\min\{g_1,g_2\}=\frac{f+g+|f-g|}2
\eeq
et
\beq
\max\{g_1,g_2\}=\frac{f+g-|f-g|}2
\eeq
Conclure alors récursivement.
%%	ÉTAPE 3
\begin{Lem}
Pour tous $x,y\in X$ et $x\neq y$, et pour tout couple de réels $\alpha,\beta$, il existe un élément $f\in\mathcal A(F)$ tel que $f(x)=\alpha$ et $f(y)=\beta$.\label{step3}
\end{Lem}
Avec les hypothèses du théorème, pour $z=x,y$, il existe $h_z\in F:h_z(x)\neq0$. Il existe également $g\in F$ tel que $g(x)\neq g(y)$. La fonction :
\beq
f:X\to\mathbb R,t\mapsto \alpha\frac{g(t)-g(y)}{g(x)-g(y)}\frac{h_x(t)}{h_x(x)}+\beta\frac{g(t)-g(x)}{g(y)-g(x)}\frac{h_y(t)}{h_y(y)}
\eeq
satisfait les hypothèses du lemme.
%%	ÉTAPE 4
\begin{Lem}
Soient $f\in C(X,\mathbb R), x_0\in X$ et $\epsilon>0$. Alors il existe $g_{x_0,\epsilon}\in\overline{\mathcal A(F)}$ :
\beq
g_{x_0,\epsilon}(x_0)=f(x_0)\text{ et }g_{x_0,\epsilon}<f+\epsilon
\eeq\label{step4}
\end{Lem}
Grâce au lemme \ref{step3}, pour chaque $y\in X\backslash\{x_0\}$, on peut choisir $h_y\in\mathcal A(F)$ telle que $f(y)=h_y(y)$ et $f(x_0)=f_y(x_0)$. Puisque $h_y$ et $f$ sont continues, l'ensemble :
\beq U_y=\{x\in X:h_y(x)<f(x)+\epsilon\}
\eeq
est un ouvert. Puisque $\forall y\in X$, $y\in U_y$, on a (l'inclusion réciproque étant explicite) :
\beq
X=\bigcup_{y\in X} U_y
\eeq
Comme $X$ est compact, il existe une sous-collection d'ouvert finie $\{U_{y_i}\}_{1\leq i\leq p}$ telle que :
\beq
X=\bigcup_{i=1}^p U_y
\eeq
Poser alors $g_{x_0,\epsilon}=\max_{i=1,...,p}h_{y_i}$. Cette fonction satisfait les hypothèses du lemme et appartient à l'algèbre engendrée par le lemme \ref{step2}.
%%	ÉTAPE 5
\begin{Lem}
Soient $f\in C(X,\mathbb R)$ et $\epsilon>0$. Il existe une fonction $g\in\overline{\mathcal A(F)}$ telle que $f-\epsilon\leq g\leq f+\epsilon$.
\end{Lem}
Par le lemme \ref{step4}, on peut choisir pour tout $x\in X$ une fonction $g_{x,\epsilon}\in\overline{\mathcal A(F)}$ telle que $g_{x,\epsilon}(x)=f(x)$ et $g_{x,\epsilon}<f+\epsilon$. Pour chaque $x\in V$, définir : 
\beq V_x=\{z\in X:f(x)-\epsilon<g_{x,\epsilon}(z)\}
\eeq
Puisque $g_{x,\epsilon}$ est définie comme le maximum de plusieurs fonctions continues, elle est continue. $V_x$ est donc ouvert pour chaque $x\in X$. Procédant comme auparavant, remarquer que l'union des membres de la famille d'ouverts vaut $X$, en extraire une sous-famille finie. Définir cette fois $g$ comme le minimum des fonctions sélectionnées. Par le lemme \ref{step2}, $g\in\overline{\mathcal A(F)}$. Elle vérifie de plus les propriétés recherchées par le lemme, et plus généralement par le théorème de Stone-Weierstrass.
\qed
\end{Pre}

\begcor
Soit $X\subset\mathbb R^n$ un ensemble compact et $F\subseteq C(X,\mathbb C)$ une famille de fonction qui sépare $X$, invariante sous conjugaison complexe et qui ne s'annule pas sur $X$. Alors l'algèbre complexe $\mathcal A_\mathbb C(F)$ engendrée par $F$ est uniformément dense dans $C(X,\mathbb C)$, c'est-à-dire $f\in\overline{\mathcal A(F)}$.
\label{cor1}

\end{Cor}
\begpre
On a $F=F^*$ car : 
\beq F^*\subseteq F=(F^*)^*\subseteq F^*\eeq Comme $F$ sépare $X$ et ne s'annule pas sur $X$, $G=(F+F^*)\cup i(F-F^*)$ ne s'annule pas sur $X$ non plus et sépare aussi $X$. Or $F\subseteq C(X,\mathbb R)$ et par le théorème de Stone-Weierstrass \ref{sw}, $C(X,\mathbb R)=\overline{\mathcal A_\mathbb R(G)}$. Comme $C(X,\mathbb C)=C(X,\mathbb R)+iC(X,\mathbb R)$ et que $\overline{\mathcal A_\mathbb R(G)},i\overline{\mathcal A_\mathbb R(G)}\subset \overline{\mathcal A_\mathbb C(G)}$, on a que $C(X,\mathbb C)=\overline{\mathcal A_\mathbb C(G)}$. Or : $\mathcal A_\mathbb C(G)=\mathcal A_\mathbb C(F)$.
\qed
\end{Pre}

\begcor
Soit $X\subset\mathbb R$ un ensemble compact. L'ensemble $\mathbb C[ X]$ est uniformément dense dans $C(X,\mathbb C)$.
\end{Cor}
\begpre
$\mathbb C[ X]=\mathcal A_\mathbb C(\{1,id_X\})
$ vérifie les hypothèses du corollaire \ref{cor1}.
\qed
\end{Pre}

\begcor
Soit $I=[a,b]$ un intervalle fermé de $\mathbb R$. L'algèbre engendrée sur les complexes par $F$ défini comme : 
\beq
F=\left\{e^{2\pi ni\frac{x-a}{b-a}},x\in I, n\in\mathbb N\right\}
\eeq
est uniformément dense dans $V=\{f:f\in C([a,b],\mathbb C),f(a)=f(b)\}$. 
\end{Cor}
\begpre
La fonction $\varphi$ définie par :
\beq
\varphi : [a,b]\to\partial B_1(0), x\mapsto e^{2\pi ni\frac{x-a}{b-a}}
\eeq
induit un homéomorphisme isométrique $\Phi:C(\partial B_1(0),\mathbb C)\to V, f\mapsto f\circ\varphi$. Or, $C(\partial B_1(0),\mathbb C)=\overline{\mathcal A_\mathbb C(\{1,z\mapsto z,z\mapsto z^*\})}$ puisque $\{1,z\mapsto z,z\mapsto z^*\}$ satisfait les hypothèses du corollaire \ref{cor1} et $F=\Phi|_{\mathcal A_\mathbb C(\{1,z\mapsto z,z\mapsto z^*\})}$.
\qed
\end{Pre}

\begin{comment}

\section{Complétion des espaces $L^p$}
\begdef
Une mesure $\mu$ définie sur une $\sigma$-algèbre $\Sigma\subseteq\mathcal P(\mathbb R^n)$ borélienne est dite intérieurement régulière si :
\beq
\forall E\in\Sigma, \mu(E)=\sup\{\mu(K):K \text{ compact et } K\subset E\}
\eeq
Elle est dite extérieurement-régulière si
\beq
\forall E\in\Sigma, \mu(E)=\inf\{\mu(V):V \text{ ouvert et } E\subset V\}
\eeq
Elle est enfin régulière si elle est simultanément extérieurement et intérieurement régulière, et localement finie si :
\beq
\forall x\in\mathbb R^n,\exists \text{ un ouvert }U\in\Sigma : x\in U\text{ et } \mu(U)<\infty
\eeq
\end{Def}

\begin{The}
Soit $\mu$ une mesure régulière et localement finie, $f\in L^p(\mathbb R^n,\mu)$ pour $1\leq p<\infty$ et $\epsilon>0$. Alors :
\beq
\exists \varphi\in C_c(\mathbb R^n):||f-\varphi||_p<\epsilon
\eeq
\end{The}

\begpre
Trouvons $\varphi\in C_c(\mathbb R^n,\mathbb R^+)$ pour une fonction $f\in L^p(\mathbb R^n,\mu)$ positive. Poser pour $m\in\mathbb N$ :
\beq
f_m:x\mapsto \chi_{B_m(0)}(x)\min\{f(x),m\}
\eeq
Alors, pour chaque $m\geq0$, $f_m\in L^1(\mathbb R^n,\mu)$. Puisque $|f-f_m|^p\leq|f|^p$, le théorème de la convergence dominée implique que $(f_m)$ converge vers $f$ dans la norme $||\cdot||_p$. Il existe donc $k\in\mathbb N$ tel que $m\geq k$ implique $||f_k-f||_p<\epsilon/2$. Par définition de l'intégrale de Lebesgue, et puisque $f_m\in L^1(\mathbb R^n,\mu)$, on peut approcher $f_m$ par une fonction étagée simple positive, celle-ci se laissant approcher par une fonction $\varphi\in C_c(\mathbb R^n,\mathbb R_+)$, telle que $||f_m-\varphi||_1<\frac{\epsilon^p}{2^pm^{p-1}}$. Puisque $f_m$ est majorée par $m$, on peut le supposer aussi pour $\varphi$. On a alors $|f_m(x)-\varphi(x)|\leq m$, et $|f_m-\varphi|^p\leq m^{p-1}|f_m-\varphi|$. Par conséquent,
\beq
||f_m-\varphi||_p^p\leq m^{p-1}||f_m-\varphi||_1\leq\left(\frac\epsilon2\right)^p
\eeq
C'est-à-dire $||f-\varphi||_p<\epsilon$.
\qed\end{Pre}

\begthe
Si $\mu$ est une mesure régulière et localement finie, l'espace de Hilbert $L^2(\mathbb R^n,\mu)$ est séparable.
\end{The}

\begpre
Soit $f\in C_c(\mathbb R^n)$. Des fonctions qui sont des combinaisons linéaires du type $\sum_{k=1}^m\alpha_k\chi_{E_k}$ pour $E_k=\times_{j=1}^n ]a_j,b_j]$ approchent $f$ uniformément dans $C_c(\mathbb R^n)$. Par conséquent, pour une mesure régulière et localement finie, les fonctions de ce type approchent $f$ dans $L^2(\mathbb R^n,\mu)$. On pourrait donc choisir les $\alpha_k,a_k,b_k$ rationnels pour approcher $f$ par une famille dénombrable.
\qed
\end{Pre}
\end{comment}
\section{Systèmes complets et orthonormés}
\begdef 
Soit $\mathcal H$ un $\mathbb K$-espace vectoriel. Une famille $F\subseteq \mathcal H$ est dite complète si l'ensemble $\text{Vect}(F)$ des combinaisons linéaires d'éléments de $F$ est un ensemble dense de $\mathcal H$.
\end{Def}

\begthe Soit $\mathcal H$ un espace de Hilbert sur $\mathbb K$ séparable. Soit $B=\{\varphi_n\}_{n\in I}$ une famille de vecteurs othonormée. Alors $|B|\leq|\mathbb N|$. \end{The}

\begpre
Puisque $\mathcal H$ est séparable, il existe une famille $D=\{\psi_n\}_{n\in \mathbb N}\subset \mathcal H$ dense dans $\mathcal H$. Il existe alors pour chaque $\varphi_k\in B$ un élément de $\psi_k\in D$ tel que $||\varphi_k-\psi_k||\leq 1/\sqrt2$. Si $\varphi_k\neq\varphi_l$, alors par orthonormalité $||\varphi_k-\varphi_l||^2=2$ est par conséquent, $||\varphi_k-\varphi_l||=\sqrt2$. Mais alors, $\psi_k\neq\psi_l$ puisque la supposition contraire entraînerait $||\varphi_k-\varphi_l||\leq||\varphi_k-\psi_k||+||\varphi_l-\psi_k||<\sqrt2$, une contradiction. On a alors une fonction injective de $B$ dans $D$.
\qed
\end{Pre}

\begthe
Soit $\mathcal H$ un espace de Hilbert séparable. Il existe alors un système complet et orthonormé $\{b_n\}_{n\in I}$ avec $I\subseteq\mathbb N$.
\end{The}

\begpre
Comme $\mathcal H$ est séparable, il existe une famille $D=\{d_n\}_{n\in\mathbb N}$ dense dans $\mathcal H$. On va itérativement construire une famille $\{v_n\}_{n\in J}\subseteq D$ de la manière suivante :
\begin{eqnarray}
\varphi(n+1)=\inf\{k\in\mathbb N:k>\varphi(n)\text{ et }d_k\notin \text{Vect}[v_0,...,v_n]\}\\
\varphi(0)=0, v_{n+1}=d_{\varphi(n+1)}
\end{eqnarray}
Clairement, la famille $\{v_n\}_{n\in\mathbb J}$ est formée de vecteurs linéairement indépendants. De plus, $\text{Vect}[v_n]_{n\in J}$ est dense dans $\mathcal H$, puisque $D\subseteq\text{Vect}[v_n]_{n\in J}$. On forme maintenant une famille de vecteurs orthonormés à partir de $\{v_n\}_{n\in J}$ et $\text{Vect}[v_n]_{n\in J}$.
\beq
b_n=\frac1{||v_n-\sum_{k=0}^{n-1}\langle b_k,v_n\rangle b_k||}\left(v_n-\sum_{k=0}^{n-1}\langle b_k,v_n\rangle b_k\right)
\eeq
Manifestement, $\{b_n\}_{n\in J}$ est une famille de vecteurs orthonormés et $\text{Vect}[b_n]_{n\in J}=\text{Vect}[v_n]_{n\in J}$, de sorte de $\{b_n\}_{n\in J}$ forme un système complet.
\qed\end{Pre}

\begthe
Soit $\{b_n\}_{n\in B}, B\subseteq\mathbb N$ une famille de vecteurs orthonormés d'un espace de Hilbert $\mathcal H$ séparable. Alors, les affirmations suivantes sont équivalentes :
\begin{enumerate}
\item $\{b_n\}_{n\in B}$ est complet
\item $\forall x\in \mathcal H$, $x=\sum_{n\in B}\langle b_n,x\rangle b_n$
\item $\forall x,y\in \mathcal H$, $\langle x,y\rangle=\sum_{n\in B}\langle x,b_n\rangle \langle b_n,y\rangle$
\end{enumerate}
\end{The}

\begpre
Montrer que $1.\implies2.$. Pour $F\subseteq B$, $|F|<\infty$, former les espaces $V_F=\text{Vect}[\{b_n\}_{n\in F}]$. Tous ces espaces sont des sous-espaces vectoriels de $\mathcal H$ fermés, complets et convexes. $\{b_n\}_{n\in B}$ étant complet, $\cup_{F\subseteq B:|F|<\infty}V_F$ est dense dans $\mathcal H$. Soient $x\in \mathcal H$, $\epsilon>0$. Il existe alors un $y\in V_F$ tel que $||x-y||<\epsilon$. A fortiori, notant $x_{V_F}$ la projection de $x$ sur $V_F$ (et de même pour $G$ ensuite), $||x-x_{V_F}||<\epsilon$, et si $F\subseteq G\subseteq B$, $|G|<\infty$, on a :
\beq
||x-x_{V_G}||=\inf\{||x-y||:y\in V_G\}\leq\inf\{||x-y||:y\in V_F\}<\epsilon
\eeq
C'est-à-dire, les projections de $x$ sur les $V_F$ convergent vers $x$. Poser $x_{V_F}=\sum_{n\in F}\lambda_n b_n$. Puisque $x-x_{V_F}\perp V_F$, on doit avoir pour $b_m\in V_F$ :
\beq
0=\langle x,b_m\rangle-\overline\lambda\langle b_m,b_m\rangle
\eeq
d'où $\lambda_m=\langle b_m,x\rangle$ et enfin :
\beq
x=\lim_{n\to\infty}\sum_{k\in B,k\leq n}\langle b_k,x\rangle b_k
\eeq
Montrer que $2.\implies3.$ 
\begin{equation}
\langle x,y\rangle=\sum_{n\in B}\sum_{m\in B}\langle b_n,x\rangle \langle y,b_m\rangle\langle b_n,b_m\rangle=\sum_{n\in B}\langle x,b_n\rangle \langle b_n,y\rangle
\end{equation}
Montrer que $3.\implies1.$. En posant $x=y$, on obtient $||x||=\sum_{n\in B}|\langle x,b_n\rangle|^2<\infty$. Ainsi, la suite $\left(\sum_{k\in B,k\leq n}\right)_{n\in\mathbb N}\langle b_k,x\rangle b_k$ est une suite de Cauchy dont on a clairement la limite $x$ car $\left(\langle x,b_k\rangle\right)_{k\in B}$ est une suite dans $l^2(B)$.
\qed
\end{Pre}
\newpage
\part{Mesures et intégrales}
\section{Séparation et partition}
Pour $A\subseteq\mathbb R^n$ et $x\in\mathbb R^n$, poser $d:\mathbb R^n\times \mathcal P(\mathbb R^n)\to\mathbb R^+$ :
\beq
(x,A)\mapsto\inf\{|x-y|:y\in A\}
\eeq
\begthe
Soit $A\subseteq\mathbb R^n$. Alors la fonction $d_A:\mathbb R^n\to\mathbb R^+$ :
\beq x\mapsto d(x,A)\eeq
est continue.
\end{The}
\begpre
Soit $A\neq\emptyset$. Soit $\epsilon>0$ et $x,y\in\mathbb R^n$ tels que $|x-y|<\delta$. Supposer (autrement inverser les rôles) que $d_A(y)\leq d_A(x)$. Utilisant l'inégalité triangulaire :
\begin{equation}
|d_A(x)-d_A(y)|=\inf_{z\in A}d(x,z)-\inf_{z\in A}d(y,z)\leq d(x,y)+\inf_{z\in A}d(y,z)-\inf_{z\in A}d(y,z)<\delta
\end{equation}
Prendre $\delta=\epsilon$.
\qed\end{Pre}
\begdef
Un ensemble de $\mathbb R^n$ est dit relativement compact si sa fermeture est compact.
\end{Def}

\begin{Lem}
Soit un compact $K\subset\mathbb R^n$. Il existe alors un ouvert $U$ relativement compact tel que $K\subset U$.\label{rc}
\end{Lem}
\begpre
Si $K=\emptyset$, prendre $U=B_1(x), x\in\mathbb R^n$. Sinon, considérer la famille d'ouverts $\{B_1(x)\}_{x\in K}$ qui recouvre $K$. Extraire une sous-famille finie $F\subset K$ qui recouvre $K$. L'ouvert :
\beq
U=\bigcup_{x\in F}B_1(x)
\eeq
satisfait aux exigences du lemme.
\qed\end{Pre}
\begdef
Soient $K\subset V\subseteq \mathbb R^n$, avec $K$ compact et $V$ ouvert. On dit qu'une fonction $f\in C_c(\mathbb R^n,[0,1])$ sépare $K$ de $\mathbb R^n\backslash V$ et note $K\prec f\prec V$, si $f^{-1}\{1\}$ est un voisinage de $K$ et si $\text{supp}(f)\subset V$.\end{Def}

\begin{Lem}\textbf{Lemme d'Urysohn}
Soient $K\subset V\subseteq\mathbb R^n$, avec $K$ compact et $V$ ouvert. Il existe alors une fonction $f$ telle que $K\prec f\prec V$.\label{ury}
\end{Lem}
\begpre
Par le lemme \ref{rc}, il existe un ouvert $U$ contenant $K$ relativement compact. Remplaçant si nécessaire $V$ par $V\cap U$, on peut supposer que $V$ est relativement compact. La fonction :
\beq g(x)=\frac{d(x,\mathbb R^n\backslash V)}{d(x,\mathbb R^n\backslash V)+d(x,K)}\eeq est manifestement définie pour tout $x\in \mathbb R^n$ et continue comme combinaison de fonctions continues. De plus, $g|_K=1$ et $g|_{\mathbb R^n\backslash V}=0$. Soient alors les ouverts $W=g^{-1}]2/3,1]$ et $U=g^{-1}]1/3,1]$. Clairement, $K\subset W\subset \overline{U}\subset V$ et la fonction :
\beq
f(x)=\frac{d(x,\mathbb R^n\backslash U)}{d(x,\mathbb R^n\backslash U)+d(x,W)}
\eeq
satisfait aux critères du lemme.
\qed
\end{Pre}

\begdef
Soit $K$ un compact de $\mathbb R^n$ et $\{V_n\}_{1\leq n\leq m}$ une collection finie d'ensembles ouverts qui recouvrent $K$. Une famille de $m$ fonctions $f_n\prec V_n$ telles que :
\beq
\sum_{n=1}^mf_n(x)=1, \forall x\in K
\eeq
est appelée une partition de $K$ subordonnée au recouvrement $\{V_n\}_{1\leq n\leq m}$.
\end{Def}

\begin{Cor}
Soit $K\subset \mathbb R^n$ compact et $\{V_n\}_{1\leq n\leq m}$ une collection finie d'ensembles ouverts qui recouvrent $K$. Il existe alors une partition de $K$ subordonnée à $\{V_n\}_{1\leq n\leq m}$.\label{partition}
\end{Cor}
\begpre
Soit $x\in K$. Il existe $V_{n_x}$ du recouvrement tel que $x\in V_n$. Par le lemme d'Urysohn \ref{ury}, il existe une fonction $g_x$ telle que $\{x\}\prec g_x\prec V_{n_x}$. L'ensemble $K_x=g_x^{-1}\{1\}$ est alors un voisinage compact de $\{x\}$. Comme $K$ est compact et puisque $\{\accentset{\circ}{K_x}\}_{x\in K}$ recouvre $K$, il existe une sous-collection finie $\{K_{x_j}\}_{j=1,...,p}$ qui recouvre $K$. Pour chaque $V_n$ du recouvrement initiale, poser :
\beq C_n=\bigcup_{K_{x_j}\subset V_n, 1\leq i\leq p}K_{x_j}\eeq Tous les $C_n$ sont compacts et leur collection recouvre $K$. De plus, $C_n\subset V_n$, $n=1,...,m$. Une nouvelle application du lemme d'Urysohn livre alors $m$ fonctions $h_n$ telles que $C_n\prec h_n\prec V_n$. Poser alors $f_1=h_1$ et $f_n=h_n\prod_{k=1}^{n-1}(1-h_k)$, pour $n\geq 2$. Clairement, $f_n\prec V_n$ pour $n=1,...,m$ et :
\beq
\sum_{n=1}^mf_n=1-\prod_{n=1}^m(1-h_n)
\eeq
De plus, si $x\in K$, $x\in C_n$ pour au moins un $n$, de sorte que $h_n(x)=1$, c'est-à-dire la propriété espérée.
\qed\end{Pre}

\section{Mesures et fonctionnelles positives}
\begdef
Une fonctionnelle $\Phi:C_c(\mathbb R^n,\mathbb R)\to\mathbb R$ est dite positive si $f\geq0$ implique $\Phi(f)\geq0$.
\end{Def}

\begdef Soit $\Phi:C_c(\mathbb R^n,\mathbb R)$ une fonctionnelle positive. Soient $K,V\subset\mathbb R^n$ avec $K$ compact, $V$ ouvert. Définir :
\begin{eqnarray}
\mu(K)=\inf\{\Phi(f):K\prec f\}\in\mathbb R^+\\
\mu(V)=\sup\{\Phi(f):f\prec V\}\in\mathbb R^+\cup\{\infty\}
\end{eqnarray}
\end{Def}

\begin{Lem}
Soient $K,V\subset\mathbb R^n$, $K$ compact, $V$ ouvert. Alors :
\begin{eqnarray}
\mu(K)=\inf\{\mu(W):K\subset W, W\text{ ouvert}\}\\
\mu(V)=\sup\{\mu(C):C\subset V, C\text{ compact}\}
\end{eqnarray}
\end{Lem}
\begpre
Si $K\subset W\subset\mathbb R^n$, $K$ compact, $W$ ouvert, alors par le lemme d'Urysohn \ref{ury}, il existe une fonction $K\prec f\prec W$. Puisque $f^{-1}\{1\}$ est un voisinage de $K$, on a :
\beq
\mu(K)=\inf\{\Phi(f):K\prec f\}\geq\inf\{\mu(U):K\subset U, U\text{ ouvert}\}\geq\mu(K)
\eeq Similairement, puisque pour une fonction $f\prec V$, on a $\text{supp}(f)\subset V$ et que $\text{supp}(f)$ est compact, on :
\beq
\mu(V)=\sup\{\Phi(f):f\prec V\}\leq\sup\{\mu(C):C\subset V, C\text{ compact}\}\leq\mu(V)
\eeq
\qed\end{Pre}

\begdef On définit une mesure intérieure $\mu_*:\mathcal P(\mathbb R^n)\to\mathbb R_+\cup\{\infty\}$ et une mesure extérieure $\mu^*:\mathcal P(\mathbb R^n)\to\mathbb R_+\cup\{\infty\}$ par :
\beq
\mu_*(E)=\sup\{\mu(K):K\subseteq E,K\text{ compact}\} \text{ et } \mu^*(E)=\inf\{\mu(V):E\subseteq V,V\text{ ouvert}\}
\eeq
\end{Def}

\begin{Lem}
Si $\{F_n\}_{n\in\mathbb N}$ est une suite de sous-ensembles de nombres réels positifs, alors :
\beq
\inf\left\{\sum_{n=0}^\infty a_n|\forall k : a_k\in F_k\right\}=\sum_{n=0}^\infty\inf F_n\in\mathbb R_+\cup\{\infty\}
\eeq\label{eqcheloue}
\end{Lem}
\begpre
Noter $A$ le terme de gauche, $B$ le terme de droite. D'abord, remarquer que $B$ minore l'ensemble $\left\{\sum_{n=0}^\infty a_n|\forall k : a_k\in F_k\right\}$ donc $B\leq A$. Si $B=\infty$, alors $A=B$. Supposer donc $B<\infty$. Alors, $\forall n\geq0$, $\inf F_n<\infty$. Soit $x>B$ et poser $\epsilon=x-B$.
\beq
\forall n\in\mathbb N, \exists a_n\in F_n:a_n<\inf F_n+\epsilon/2^{n+1}
\eeq
Alors, $A\leq\sum_{n\geq0}a_n\leq B+\epsilon=x$. Puisque $x$ est arbitraire, $A\leq B$.
\qed\end{Pre}

\begin{Lem}
La mesure intérieure $\mu_*$ est sur-additive alors que la mesure extérieure est sous-additive. C'est-à-dire que si $\{E_n\}_{n\in\mathbb N}$ est une suite de sous-ensembles de $\mathbb R^n$ deux à deux disjoints, on a :
\beq
\mu_*\left(\bigcup_{n=0}^\infty E_n\right)\geq\sum_{n=0}^\infty \mu_*(E_n)
\eeq
et
\beq
\mu^*\left(\bigcup_{n=0}^\infty E_n\right)\leq\sum_{n=0}^\infty \mu^*(E_n)
\eeq\label{sursous}
\end{Lem}

\begpre
Pour montrer la première inégalité, il suffit de montrer que pour tout $m\in\mathbb N$ :
\beq
\mu_*\left(\bigcup_{n=0}^m E_n\right)\geq\sum_{n=0}^m \mu_*(E_n)
\eeq
Par définition de la mesure extérieure,
\beq
\sum_{n=0}^m\mu_*(E_n)=\sup\left\{\sum_{n=0}^m\mu(K_n):K_n\subseteq E_n, K_n\text{ compact}\right\}
\eeq
Puisque l'union finie de compact est contenue dans l'union finie des $(E_n)_{n=0}^m$, il suffit de montrer que :
\beq
\mu\left(\bigcup_{n=0}^mK_n\right)\geq\sum_{n=0}^m\mu(K_n)
\eeq
Plus simplement, il suffit de montrer que pour deux compacts disjoints $K_1$ et $K_2$, $\mu(K_1\cup K_2)\geq \mu(K_1)+ \mu(K_2)$. Soit alors $K_1\cup K_2\prec f$. Comme $K_1$ est disjoint de $K_2$, le lemme d'Urysohn \ref{ury} assure qu'il existe $f_1$ telle que $K_1\prec f_1\prec\mathbb R^N\backslash K_2$. Par le même argument, il existe $f_2$ telle que $K_2\prec f_2\prec \mathbb R^N\backslash\text{supp}(f_1)$. Alors $K_1\cup K_2\prec f(f_1+f_2)\leq f$, $K_1\prec ff_1$ et $K_2\prec ff_2$, de sorte que :
\beq
\Phi(f)\geq\Phi(ff_1)+\Phi(ff_2)\geq\mu(K_1)+\mu(K_2)
\eeq
En prenant l'infimum sur toutes les fonctions $f$, on trouve $\mu(K_1\cup K_2)\geq \mu(K_1)+ \mu(K_2)$.

Soit maintenant une suite d'ouverts $\{V_n\}_{n\in\mathbb N}$, $E_n\subseteq V_n$. Clairement, $\bigcup_{n=0}^\infty E_n\subseteq\bigcup_{n=0}^\infty V_n$. Soit $f\prec \bigcup_{n=0}^\infty V_n$. Puisque $f$ est à support compact, le corollaire \ref{partition} implique qu'il existe une partition $(f_n)_{n=0}^m$ de $\text{supp}(f)$ subordonnée au recouvrement fini $\{V_n\}_{n=0}^m$. Alors :
\beq
\Phi(f)=\Phi\left(f\sum_{n=0}^mf_n\right)=\sum_{n=0}^m\Phi(ff_n)\leq\sum_{n=0}^m\mu(V_n)\leq\sum_{n=0}^\infty \mu(V_n)
\eeq
Il suit en prenant l'infimum à gauche :
\beq
\mu\left(\bigcup_{n=0}^\infty V_n\right)\leq\sum_{n=0}^\infty\mu(V_n)\eeq
Par le lemme \ref{eqcheloue} :
\beq
\mu^*\left(\bigcup_{n=0}^mE_n\right)\leq\inf\left\{\mu\left(\bigcup_{n=0}^mV_n\right):\forall k, E_k\subseteq V_k\right\}\leq\inf\left\{\sum_{n=0}^\infty \mu(V_n):\forall k, E_k\subseteq V_k\right\}=\sum_{n=0}^\infty\mu^*(E_n)
\eeq
\qed\end{Pre}

\begdef
Soit $\Phi: C_c(\mathbb R^N,\mathbb R)\to\mathbb R$ une fonctionnelle positive ainsi que ses mesures intérieures et extérieures $\mu_*$ et $\mu^*$. Un ensemble $E\subseteq\mathbb R^N$ est dit mesurable si et seulement si pour tout compact $K\subset \mathbb R^N$ :
\beq
\mu_*(K\cap E)=\mu^*(K\cap E)
\eeq
La collection des ensembles mesurables est dénotée $\Sigma$.
\end{Def}

\begin{Lem}
Soit $\{E_n\}_{n\in\mathbb N}\subseteq\Sigma$ une suite d'ensembles mesurables deux à deux disjoints. Alors $\bigcup_{n=0}^\infty E_n\in\Sigma$ et :
\beq
\mu_*\left(\bigcup_{n=0}^\infty E_n\right)=\sum_{n=0}^\infty\mu_*(E_n)
\eeq \label{addisub}
\end{Lem}

\begpre
Commencer par montrer que l'union des $\{E_n\}$ est dans $\Sigma$. Prendre $K\subset\mathbb R^n$ compact. Alors :
\begin{equation}\begin{split}
\mu_*\left(K\cap\bigcup_{n\in\mathbb N}E_n\right)&=\mu_*\left(\bigcup_{n\in\mathbb N}K\cap E_n\right)
\\& \leq\mu^*\left(\bigcup_{n\in\mathbb N}K\cap E_n\right)\text{, car $\mu_*\leq\mu^*$}
\\& \leq \sum_{n\in\mathbb N}\mu^*(K\cap E_n)\text{, par sous-additivité \ref{sursous}}
\\& =\sum_{n\in\mathbb N}\mu_*(K\cap E_n)\text{, car $E_n\in\Sigma$}
\\& \leq \sum_{n\in\mathbb N}\mu_*(E_n)\text{, par monotonie}
\\& \leq \mu_*\left(\bigcup_{n\in\mathbb N}E_n\right)\text{, par sur-additivité \ref{sursous}}
\end{split}\end{equation}
Ainsi, d'une part : 
\beq
\mu_*\left(K\cap\bigcup_{n\in\mathbb N}E_n\right)\leq\mu^*\left(K\cap\bigcup_{n\in\mathbb N}E_n\right)\leq\sum_{n\in\mathbb N}\mu_*(K\cap E_n)\leq\mu_*\left(K\cap\bigcup_{n\in\mathbb N}E_n\right)
\eeq
donc $\bigcup_{n\in\mathbb N}E_n\in\Sigma$. D'autre part, en prenant le supremum sur tous les compacts $K\subseteq\bigcup_{n\in\mathbb N}E_n$ :
\beq
\mu_*\left(\bigcup_{n\in\mathbb N}E_n\right)\leq \sum_{n\in\mathbb N}\mu_*(E_n)\leq\mu_*\left(\bigcup_{n\in\mathbb N}E_n\right)
\eeq
\qed\end{Pre}

\begin{Lem} Soient $E,F\in\Sigma$, alors $E\backslash F\in \Sigma$.\label{comple}
\end{Lem}
\begpre
Soit $K\subset\mathbb R^N$ compact. Il reste à montrer que $\mu_*(K\cap(E\backslash F))\geq\mu^*(K\cap(E\backslash F))$. Comme $K\cap(E\backslash F)\subseteq K$, $K\cap E\subseteq K$ et $K\cap F\subseteq K$, il existe des ouverts $V_E$ et $V_F$ de mesures extérieures finies tels que $K\cap E\subseteq V_E$ et $K\cap F\subseteq V_F$. Soient des compacts $K\subseteq K\cap E$ et $K_F\subseteq K\cap F$. On a alors :
\beq
K_E\backslash V_F\subseteq K\cap E\backslash K\cap F=K\cap(E\backslash F)\subseteq V_E\backslash K_F
\eeq
Remarquer alors que $K_E\backslash V_F$ est compact, $V_E\backslash K_F$ est ouvert et que :
\beq
V_E\backslash K_F=(V_E\backslash K_E)\cup (K_E\backslash V_F) \cup (V_F\backslash K_F)
\eeq
Par monotonie et sous-additivité de la mesure extérieure, on a :
\begin{equation}\begin{split}
\mu^*(K\cap(E\backslash F))&\leq\mu^*(V_E\backslash K_F)\\ &\leq \mu^*(V_E\backslash K_E)+ \mu^*(K_E\backslash V_F) + \mu^*(V_F\backslash K_F) \\ &\leq \mu^*(V_E)-\mu^*(K_E)+\mu_*(K\cap(E\backslash F))+\mu^*(V_F)-\mu^*(K_F)
\end{split}\end{equation}
en utilisant la mesurabilité des compacts. Avec la mesurabilité de $K\cap E$ et $K\cap F$, on obtient de plus :
\beq
\inf\{\mu^*(V_E)-\mu^*(K_E)+\mu^*(V_F)-\mu^*(K_F):K_E\subseteq K\cap E\subseteq V_E, K_F\subseteq K\cap F\subseteq V_F\}=0
\eeq
\qed\end{Pre}

\begin{The}
Soit $\Phi:C_c(\mathbb R^N,\mathbb R))\to\mathbb R$ une fonctionnelle positive, sa mesure intérieure $\mu_*$ et extérieure $\mu^*$ ainsi sur les ensembles mesurables $\Sigma$. Alors $\Sigma$ est une $\sigma$-algèbre borelienne. De plus $\mu$ définit alors une mesure régulière et complète sur $\Sigma$.
\end{The}

\begpre
Clairement, $\emptyset\in\Sigma$. Le complémentaire d'une ensemble mesurable est mesurable par le lemme \ref{comple}. Soit maintenant $\{E_n\}_{n\in\mathbb N}\subset\Sigma$. Poser $F_0=E_0$ et pour $n\geq1$, $F_n=E_n\backslash(\bigcup_{0\geq m\geq n-1}F_m)$. $\{F_n\}_{n\in\mathbb N}\subset \Sigma$ par le lemme \ref{addisub} et $\bigcup_{n\in\mathbb N}E_n=\bigcup_{n\in\mathbb N}E_n$. Ainsi, $\Sigma$ est une $\sigma$-algèbre. Ensuite, puisque les fermés sont mesurables, les ouverts le sont aussi par le lemme \ref{comple}, c'est-à-dire $\Sigma$ est borelienne. Montrer maintenant que $\mu$ définit bien une mesure sur $\Sigma$. Soit $E\in\Sigma$. Par sous-additivité de la mesure intérieure, la mesurabilité de E et des $B_n(0)$ ainsi que la sur-additivité de $\mu_*$, on a :
\begin{equation}\begin{split}
\mu^*(E)&=\mu^*\left(\left(\bigcup_{n\in\mathbb N^*}B_n(0)\right) \cap E\right)\\
&=\mu^*\left(\left(\bigcup_{n\in\mathbb N^*}B_n(0)\backslash B_{n-1}(0) \right) \cap E\right)
\\& \leq\sum_{n\in\mathbb N^*}\mu^*(B_n(0)\backslash B_{n-1}(0) \cap E)
\\& =\sum_{n\in\mathbb N^*}\mu^*((B_n(0)\cap(B_n(0)\backslash B_{n-1}(0) \cap E))
\\& = \sum_{n\in\mathbb N^*}\mu_*((B_n(0)\cap(B_n(0)\backslash B_{n-1}(0) \cap E))
\\& = \sum_{n\in\mathbb N^*}\mu_*(B_n(0)\backslash B_{n-1}(0) \cap E)
\\&=\mu_*\left(\left(\bigcup_{n\in\mathbb N^*}B_n(0)\backslash B_{n-1}(0) \right) \cap E\right)
\\& = \mu_*(E)
\end{split}\end{equation}
En utilisant le lemme \ref{addisub}, il est clair que $\mu$ définit bien une mesure. La régularité de $\mu$ est claire par les dernières équations, et la complétude découle de la définition de $\mu$.
\qed\end{Pre}

\begdef
Si $\Phi$ est l'intégrale de Riemann, alors $\mu$ est la mesure de Lebesgue.
\end{Def}

\begin{The}
\textbf{Théorème de Riesz-Kakutani} Soit $\Phi:C_c(\mathbb R^n)\to\mathbb R$ une fonctionnelle positive. Il existe alors une mesure $\mu$ régulière et complète, définie sur une $\sigma$-algèbre $\Sigma\subseteq\mathcal P(\mathbb R^n)$ borelienne, telle que :
\beq
\forall f\in C_c(\mathbb R^n) : \Phi(f)=\int_{\mathbb R^n}f\mathrm{d}\mu\eeq\label{rk}
\end{The}

\begpre
Il est suffisant de prouver le théorème dans le cas où $f$ prend des valeurs réelles. En fait, il suffit de prouver que :
\beq
\forall f\in C_c(\mathbb R^N,\mathbb R) : \Phi f\leq \int_{\mathbb R^N}f\mathrm{d}\mu
\eeq
En effet, dès lors :
\beq
-\Phi f=\Phi(-f)\leq\int_{\mathbb R^N}(-f)\mathrm{d}\mu=-\int_{\mathbb R^N}f\mathrm{d}\mu
\eeq
Soit $K=\text{supp}(f)$ et $[a,b]$ son image. Soit $\epsilon>0$ et pour chaque $i=0,...,n$ avec $y_i-y_{i-1}<\epsilon$ :
\beq
y_0<a<y_1<...<y_n=b
\eeq
Définir pour chaque $i=1,..,n$ :
\beq
E_i=\{x\in\mathbb R^N:y_{i-1}<f(x)<y_i\}\cap K
\eeq
Car $f$ est continue, $f$ est Borel-mesurable et les ensembles $E_i$ sont des ensembles de Borel disjoints d'union $K$. Aussi, il existe des ouverts $V_i$ tels que $E_i\subseteq V_i$ et $\mu(V_i)<\mu(E_i)+\epsilon/n$. De plus :
\beq \forall x\in V_i: f(x)<y_i+\epsilon\eeq
Par le corollaire \ref{partition}, il existe, pour chaque $i$, $h_i\prec V_i$ une partition de $K$ extraite de la famille finie des $\{V_i\}_{i=1}^n$. En particulier, $f=\sum h_i f$ d'où :
\beq
\mu(K)\leq\Phi\left(\sum_{i=1}^n h_i\right)=\sum_{i=1}^n \Phi h_i
\eeq
D'autre part, puisque $h_if\leq(y_i+\epsilon)h_i$ et $y_i<f(x)+\epsilon$ sur $E_i$ :
\begin{equation}\begin{split}
\Phi f&=\sum_{i=1}^n\Phi(h_if)
\leq\sum_{i=1}^n(y_i+\epsilon)\Phi h_i
= \sum_{i=1}^n(|a|+y_i+\epsilon)\Phi h_i-|a|\sum_{i=1}^n\Phi h_i
\leq \sum_{i=1}^n(|a|+y_i+\epsilon)(\mu(E_i)+\epsilon/n)-|a|\mu(K)
\\ &= \sum_{i=1}^n(y_i-\epsilon)\mu(E_i)+2\epsilon\mu(K)+\frac{\epsilon}n\sum_{i=1}^n(|a|+y_i+\epsilon)
\leq \int_{\mathbb R^N}f\mathrm{d}\mu+\epsilon(2\mu(K)+|a|+b+\epsilon)
\end{split}\end{equation}
\qed\end{Pre}

\section{Résultats de densité}

\begdef
Une mesure $\mu$ définie sur une $\sigma$-algèbre $\Sigma\subseteq\mathcal P(\mathbb R^n)$ borélienne est dite intérieurement régulière si :
\beq
\forall E\in\Sigma, \mu(E)=\sup\{\mu(K):K \text{ compact et } K\subset E\}
\eeq
Elle est dite extérieurement-régulière si
\beq
\forall E\in\Sigma, \mu(E)=\inf\{\mu(V):V \text{ ouvert et } E\subset V\}
\eeq
Elle est enfin régulière si elle est simultanément extérieurement et intérieurement régulière, et localement finie si :
\beq
\forall x\in\mathbb R^n,\exists \text{ un ouvert }U\in\Sigma : x\in U\text{ et } \mu(U)<\infty
\eeq
\end{Def}

\begin{Lem}
Une mesure régulière et localement finie $\mu$ dans $\mathbb R^n$ assigne à tout compact $K$ de $\mathbb R^n$ une mesure finie.
\end{Lem}
\begin{Pre}
Pour chaque $x\in K$ il existe un ouvert $V_x$ contenant $x$ et de mesure finie. L'ensemble des tels $V_x$ recouvre $K$, on peut donc en extraire un sous-recouvrement fini $(V_i)_{i=1,..p}$, associé aux $(x_i)_{i=1,..p}$. Poser :
\beq
E_i=K\bigcap_{i=1}^p\left(V_i\backslash \bigcup_{j=1}^{i-1} V_j\right)
\eeq
La régularité de $\mu$ implique la mesurabilité de ces ensembles. De plus, $K=\bigcup_{i=1}^pE_i$ et cette union est disjointe. Ainsi :
\beq
\mu(K)=\sum_{i=1}^p\mu(E_i)\leq\sum_{i=1}^p\mu(V_i)<\infty
\eeq
\qed\end{Pre}

\begin{Lem}
L'espace des fonctions simples Lebesgue intégrables est dense dans $L^p(\mathbb R^n,\mu)$, pour une mesure régulière et localement finie $\mu$.
\label{densite1}\end{Lem}

\begin{Pre}
Trouvons $\varphi\in C_c(\mathbb R^n,\mathbb R^+)$ pour une fonction $f\in L^p(\mathbb R^n,\mu)$ positive. Poser pour $m\in\mathbb N$ :
\beq
f_m:x\mapsto \chi_{B_m(0)}(x)\min\{f(x),m\}
\eeq
Alors, pour chaque $m\geq0$, $f_m\in L^1(\mathbb R^n,\mu)$. Puisque $|f-f_m|^p\leq|f|^p$, le théorème de la convergence dominée implique que $(f_m)$ converge vers $f$ dans la norme $||\cdot||_p$. Il existe donc $k\in\mathbb N$ tel que $m\geq k$ implique $||f_k-f||_p<\epsilon/2$. Par définition de l'intégrale de Lebesgue, et puisque $f_m\in L^1(\mathbb R^n,\mu)$, on peut approcher $f_m$ par une fonction étagée simple positive, celle-ci se laissant approcher par une fonction $\varphi\in C_c(\mathbb R^n,\mathbb R_+)$, telle que $||f_m-\varphi||_1<\frac{\epsilon^p}{2^pm^{p-1}}$. Puisque $f_m$ est majorée par $m$, on peut le supposer aussi pour $\varphi$. On a alors $|f_m(x)-\varphi(x)|\leq m$, et $|f_m-\varphi|^p\leq m^{p-1}|f_m-\varphi|$. Par conséquent,
\beq
||f_m-\varphi||_p^p\leq m^{p-1}||f_m-\varphi||_1\leq\left(\frac\epsilon2\right)^p
\eeq
C'est-à-dire $||f-\varphi||_p<\epsilon$.
\qed\end{Pre}

\begin{Lem}
L'espace des combinaisons linéaires de fonctions indicatrices sur des ouverts de mesures finies ou sur des compacts est dense dans $L^p(\mathbb R^n,\mu)$, pour une mesure régulière et localement finie $\mu$.\label{densite2}
\end{Lem}

\begin{Pre}
Soient $\epsilon>0$ et $E$ un ensemble mesurable de $\mathbb R^n$ et de mesure finie. Puisque $\mu$ est régulière, il existe un compact $K_E$ et un ouvert $V_E$ tel que $K_E\subseteq E\subseteq V_E$ et :$$\mu(V_E)-\epsilon^p\leq\mu(E)\leq\mu(K_E)+\epsilon^p$$. Il suit :
\beq
||\chi_{V_E}-\chi_{E}||_p,||\chi_{K_E}-\chi_{E}||_p<\epsilon
\eeq
Utiliser le lemme \ref{densite1} pour conclure.
\qed\end{Pre}

\begin{Lem}
Pour une mesure régulière et localement finie $\mu$, l'espace des fonctions continues à support compact est dense dans $L^p(\mathbb R^n,\mu)$ :
\beq \overline{C_c(\mathbb R^n)}^{||\cdot||_{L^p(\mathbb R^n,\mu)}}=L^p(\mathbb R^n,\mu)\eeq\label{densite3}
\end{Lem}

\begin{Pre}
Grâce au lemme \ref{densite2}, on peut se concentrer sur les indicatrices d'ouverts de mesure finie ou de compact. Par le lemme d'Urysohn \ref{ury}, il existe $f\in C_c(\mathbb R^n)$ tel que $K\prec f\prec V$. Par monotonie de l'intégrale de Lebesgue, on a dans le cas où $\mu^K+\epsilon^p\geq\mu(V)$ :
\beq
||\chi_K-f||_1, ||\chi_V-f||_1\leq \epsilon^p
\eeq
ce qui implique :
\beq||\chi_K-f||_p, ||\chi_V-f||_1\leq \epsilon\eeq
\qed\end{Pre}

\begin{The} 
Pour une mesure régulière et localement finie $\mu$, l'espace des fonctions continues à support compact et infiniment continument dérivable est dense dans $L^p(\mathbb R^n,\mu)$ :
\beq \overline{C_c^\infty(\mathbb R^n)}^{||\cdot||_{L^p(\mathbb R^n,\mu)}}=L^p(\mathbb R^n,\mu)\eeq
\label{densite_smooth}\end{The}

\begin{Pre}
Partir du lemme \ref{densite3}. Un compact dans $\mathbb R^n$ étant fermé et borné, il doit pour une fonction $f\in C_c(\mathbb R^n)$ y avoir $L>0$ tel que $\text{supp}(f)\subseteq[-L,L]^n$. La fonction étant uniformément continue sur ce dernier ensemble, il doit exister une fonction simple du type :
\beq
s=\sum_{l=1}^ms_l\chi_{d_n+[0,p[^n}
\eeq
telle que :
\beq
|f-s|<\frac{\epsilon}{\mu\left([-L,L]^n\right)^{1/p}}\implies||f-s||_p<p
\eeq
Finalement, pour un intervalle $[a,b[^n$, considérons les fonctions lisses du type $(f_n)=\left(\prod_{k=1}^mf_{m,k})\right)$ où $ m\in\mathbb N^*$ et 
\beq
f_{m,k}(x_k)=\chi_{]a-1/m,b[}(x_k)e^\frac1{m^2\left(x_k+1/m-a\right)\left(x_k-b\right)}
\eeq
Alors, $\lim_{m\to\infty}f_m(x)=\chi_{[a,b[^n}(x)$ simplement et chaque élément de la suite est intégrable. Par le théorème de la convergence dominée, la suite tend dans $L^p(\mathbb R^n)$ vers $\chi_{[a,b]^n}$.
\qed\end{Pre}

\begin{Cor}
Si $\lambda$ est la mesure de Lebesgue, alors l'espace de Schwartz est dans $L^p(\mathbb R^n)$ :
\beq \overline{\mathscr S(\mathbb R^n)}^{||\cdot||_{L^p(\mathbb R^n,\lambda)}}=L^p(\mathbb R^n,\lambda)\eeq
\end{Cor}



\newpage
\part{Opérateurs bornés}
\begin{Def}
Soient $X,Y$ des espaces vectoriels normés et soit $A\in\mathcal L(X,Y)$. La transposée $A^{T}\in\mathcal L(Y^*,X^*)$ de $A$ est définie par :
\beq
A^T(\eta)=\eta\circ A
\eeq
\end{Def}

Noter $\Gamma$ l'application du théorème de Riesz-Fréchet l'isomorphisme anti-linéaire isométrique associant à chaque élément $x\in\mathcal H$ la fonctionnelle linéaire et continue de $\mathcal H'$ définie par $x^*(y)=\langle y,x\rangle$.

\begin{Def}
Soit $A\in\mathcal L(\mathcal H)$. L'adjoint de $A$ est défini par :
\beq
A^*=\Gamma^{-1}\circ A^T\circ \Gamma
\eeq
\end{Def}

\section{Spectre des opérateurs bornés}

L'ensemble des opérateurs de $\mathcal L(\mathcal H)$, dits Hilbertiens, qui sont inversibles est noté $\text{Inv}(\mathcal L(\mathcal H))$ ou $\text{Inv}(\mathcal H)$.

\begin{The}
$\text{Inv}(\mathcal L(\mathcal H))$ est un ouvert de $\mathcal L(\mathcal H)$.
\label{invouvert}
\end{The}
\begpre
Soient $A\in\text{Inv}(\mathcal H)$, $B\in B_{||A^{-1}||}(A)$. Alors $||BA^{-1}-1||\leq||B-1||||A^{-1}||<1$, c'est-à-dire $BA^{-1}\in\text{Inv}(\mathcal H)$ et à fortiori $B$ inversible.
\qed\end{Pre}

\begin{Lem}
Si $A\in\mathcal L(\mathcal H)$, alors $\sigma(A)$ est fermé et sous-ensemble du disque centré à l'origine de $\mathbb C$ de rayon $||A||$.
\label{rayleqnorm}\end{Lem}
\begin{Pre}
Supposer que $|\lambda|>||A||$, alors $||A\lambda^{-1}||<1$ et $1-\lambda^{-1}A\in\text{Inv}(\mathcal H)$. Par conséquent, $\lambda-A\in\text{Inv}(\mathcal H)$ et $\lambda\notin\sigma(A)$. Si $\lambda\notin\sigma(A)$ et $|\mu-\lambda|<||(A-\lambda)^{-1}||$, alors par le théorème \ref{invouvert}, $A-\mu\in\text{Inv}(\mathcal H)$. C'est-à-dire le complément spectre d'un opérateur est ouvert.
\qed\end{Pre}

\begin{Def}
Le rayon spectral d'un opérateur $A\in\mathcal L(\mathcal H)$ est défini par 
\beq
r(A)=\sup\{||z||:z\in\sigma(A)\}
\eeq
\end{Def}

\begin{The} Le spectre d'opérateurs bornés vérifie :
\begin{enumerate}
\item Si $U\in\mathcal L(\mathcal H)$ est unitaire, alors $\sigma(U)\subseteq \partial B_1(0)$.
\item Si $A\in\mathcal L(\mathcal H)$ est auto-adjoint, alors $\sigma(A)\subseteq [-||A||,||A||]\subset \mathbb R$ et $r(A)=||A||$.
\item Si $B\in\mathcal L(\mathcal H)$ et $p$ est un polynômes à coefficients $(a_k)_{0\leq k\leq n}$ complexes, alors :
\beq
\sigma(p(B))=p(\sigma(B))
\eeq
\end{enumerate}
\end{The}

\begin{Pre}
\begin{enumerate}
\item Soit $\lambda\in\mathbb C$ tel que $|\lambda|>1$. Alors $U-\lambda=\lambda\left(\frac U\lambda-1\right)$ et $\left|\left|\frac U\lambda\right|\right|<1$. Donc l'inverse de $\left(\frac U\lambda-1\right)$ existe. Si $\lambda\in B_1(0)\backslash \{0\}$, on a en utilisant que $U^*$ est unitaire que $\lambda U^*<1$, que $U-\lambda=U(1-\lambda U^*)$ inversible. Si $\lambda=0$, $U-\lambda=U^*$.
\item Le lemme \ref{rayleqnorm} assure $r(A)\leq||A||$. Montrer désormais que le spectre est réel. En effet, l'opérateur $\exp(iA)$ est unitaire. En utilisant le premier résultat, on conclut que le spectre est réel. Ensuite, supposer par l'absurde que ni $||A||$ ni $-||A||$ ne fasse partie du spectre. Dans ce cas, $(A+||A||)(A-||A||)=A^2-||A||^2\in\text{Inv}{\mathcal L(\mathcal H)}$. Il existe donc $B\in\mathcal L(\mathcal H)$ inverse de cet opérateur. De plus, 
\beq
(\cdot,\cdot):\mathcal H\times\mathcal H\to\mathbb C, (x,y)\mapsto \langle x,(||A||^2-A^2)y\rangle
\eeq 
est une application sesqui-linéaire positive donc vérifie Cauchy-Schwartz. Par définition de la norme, il existe une suite $(x_n)\in\mathcal H$ de vecteurs unitaires telle que $||A||=\lim_{n\to\infty}||Ax_n||$. Alors, en utilisant que $A$ est auto-adjoint :
\beq\begin{split}
1&=||x_n||^2=\langle (A^2-||A||^2)Bx_n,x_n\rangle=\left|\langle Bx_n,(||A||^2-A^2)x_n\rangle\right|\\&\leq \langle Bx_n,(||A||^2-A^2)Bx_n\rangle^{1/2}\langle x_n,(||A||^2-A^2)x_n\rangle^{1/2}\leq ||B||^{1/2}(||A||^2-||Ax_n||^2)^{1/2}
\end{split}\eeq Ce qui est absurde. Ainsi, $||A||\in\sigma(A)$ ou $-||A||\in\sigma(A)$.
\item Pour $\lambda\in\mathbb C$, $p(Z)-\lambda\in\mathbb C[Z]$. Par le théorème fondamental de l'algèbre, il existe $n$ nombres complexes tels que $p(Z)-\lambda=a\prod_{j=1}^n(Z-\lambda_j)$, $a\in\mathbb C$. Il suit :
\beq
\begin{split}
\lambda\in\sigma(p(B))\Leftrightarrow\exists k:\lambda_k\in\sigma(B)\Leftrightarrow\exists \lambda_k\in\sigma(B):p(\lambda_k)-\lambda=0\Leftrightarrow\lambda\in p(\sigma(B))
\end{split}
\eeq
\end{enumerate}
\qed\end{Pre}

\section{Le calcul fonctionnel}
\begin{Def} Soit $A$ un opérateur borné auto-adjoint sur un espace de Hilbert et $f\in C(\sigma(A),\mathbb R)$. Définir :
\beq
f(A)=\lim_{n\to\infty} p_n(A)
\eeq
pour $(p_n)$ une suite de polynôme convergeant uniformément sur $\sigma(A)$ vers $f$.
\label{conti}
\end{Def}

\begin{Prop} La définition \ref{conti} définit uniquement toutes les fonctions continues du spectre de $A$, borné et auto-adjoint, vers $\mathbb R$.
\end{Prop}
\begin{Pre}
D'après le théorème de Stone-Weierstrass \ref{sw}, il existe une suite $(p_n)_{n\in\mathbb N}$ de polynômes à coefficients réels qui converge uniformément sur $\sigma(A)$ vers $f$. Poser la suite $(p_n(A))_{n\in\mathbb N}$ d'opérateurs auto-adjoints dans $\mathcal L(\mathcal H)$. Puisque $\sigma(p_n(A))=p_n(\sigma(A))$ :
\beq
||p_n(A)||_{\mathcal L(\mathcal H)}=r(p_n(A))=\sup\{||\lambda||:\lambda\in\sigma(p_n(A))\}=\sup\{||\lambda||:\lambda\in p_n(\sigma(A))\}=||p_n||_{L^\infty(\sigma(A))}
\eeq
Par conséquence, puisque $(p_n)$ est de Cauchy pour $||\cdot||_{L^\infty(\sigma(A))}$, $(p_n(A))$ est de Cauchy pour $||\cdot||_{\mathcal L(\mathcal H)}$ et converge donc vers un élément dans ${\mathcal L(\mathcal H)}$, qui est par définition $f(A)$. Montrer que cette définition ne dépend pas de la suite de polynômes. Soit $(q_n)$ une autre suite de polynômes convergeant uniformément sur $\sigma(A)$ vers $f$. Alors :
\beq
|| f(A)-q_n(A)||\leq ||f(A)-p_n(A)||+ ||p_n(A)-q_n(A)||\leq ||f(A)-p_n(A)||+||p_n-f||_{L^\infty(\sigma(A))}+||q_n-f||_{L^\infty(\sigma(A))}
\eeq d'où l'unicité de la définition.
\qed\end{Pre}

\begin{The} Soit $A\in\mathcal L(\mathcal H)$ un opérateur auto-adjoint. Il existe alors un unique $^*$-morphisme unitaire et isométrique $\Phi:C(\sigma(A))\to\mathcal L(\mathcal H)$ avec les propriétés suivantes :
\begin{enumerate}
\item $\Phi(x\mapsto1)=1$ et $\Phi(x\mapsto x)=A$
\item $\forall f,g\in C(\sigma(A)), \lambda\in\mathbb C$:
\beq
\Phi(f+\lambda g)=\Phi(f)+\lambda\Phi(g), \Phi(fg)=\Phi(f)\Phi(g), \Phi(f)^*=\Phi\left(\overline f\right)
\eeq
\item $\forall f\in C(\sigma(A),\mathbb C)$, $||\Phi(f)||=||f||_{L^\infty(\sigma(A))}$
\end{enumerate}
\end{The}
\begin{Pre}
Prendre $\Phi(f)=f(A)$. Seul l'unicité demande une preuve explicite. Soit $\Psi$ un tel morphisme. Pour tout polynôme $p\in\mathbb C[\sigma(A)]$, $\Psi(p)=p(A)$. Comme ce morphisme est supposé isométrique et que l'ensemble des polynômes est dense dans $\{f(A):f\in C(\sigma(A))\}$ pour la norme opérateur, on peut conclure $\Psi(f)=f(A)$, $\forall f\in C(\sigma(A))$.
\qed\end{Pre}

\section{Décomposition Spectrale}
\begin{The} Soient $\mathcal H$ un espace de Hilbert séparable et $A\in\mathcal L(\mathcal H)$ un opérateur auto-adjoint. Il existe alors une famille de vecteurs orthonormés $\{e_k:k\in K\}$, $|K|\leq|\mathbb N|$, telle que :
\beq
\mathcal H=\bigoplus_{k\in K}\overline{\{f(A)e_k:f\in C(\sigma(A))\}}
\eeq\label{decomph}
\end{The}
\begpre Soit $(b_n)$ une famille orthonormée, complète et dénombrable de $\mathcal H$. Considérer :
\beq
H_0 = \overline{\{f(A)b_0:f\in C(\sigma(A))\}}
\eeq
Il est invariant par calcul fonctionnel et fermé. Définir $j_1=\min\{n\in\mathbb N^*:b_n\notin H_0\}$. Les $b_i$ d'indice plus petits sont clairement toujours dans $H_0$ et $(1-P_{H_0})b_{j_1}\neq0$. Normaliser cette composante orthogonale et lui appliquer le calcul fonctionnel pour définir :
\beq
H_1=\left\{f(A)\frac{(1-P_{H_0})b_{j_1}}{||(1-P_{H_0})b_{j_1}||}:f\in C(\sigma(A)))\right\}^{\perp \perp}
\eeq
Les deux espaces sont alors en somme direct et leur somme directe et un sous-espace vectoriel fermé de l'espace de Hilbert. Ils sont de plus invariants par calcul fonctionnel. Définir par récurrence $j_{n+1}=\min\{m\in\mathbb N^*, m\geq 1+j_n:b_m\notin \oplus_{k=0}^nH_k\}$ et 
\beq
H_{n+1}=\left\{f(A)\frac{(1-P_{\oplus_{k=0}^nH_k})b_{j_{n+1}}}{||(1-P_{\oplus_{k=0}^nH_k})b_{j_{n+1}}||}:f\in C(\sigma(A))||\right\}^{\perp \perp}
\eeq
C'est un sous-espace fermé de $\mathcal H$ contenant par construction tous ses éléments. Conclusion établie.
\qed\end{Pre}

\begin{The}\textbf{Décomposition spectrale des opérateurs auto-adjoints bornés} Soient $\mathcal H$ un espace de Hilbert séparable et $A\in\mathcal L(\mathcal H)$ un opérateur auto-adjoint. Il existe alors une famille d'indice $K$ de cardinalité au plus dénombrable, une famille de mesures boréliennes et régulières $\{\mu_k\}_{k\in K}$ sur $\sigma(A)$ et un isomorphime unitaire $V:\mathcal H\to\bigoplus_{k\in K}L^2(\sigma(A),\mu_k)$ tels que 
\beq
VAV^{-1}=M_x
\eeq
où $M_x$ est l'opérateur de multiplication par $x$ sur $\bigoplus_{k\in K}L^2(\sigma(A),\mu_k)$.
\label{decompoborne}\end{The}
\begpre
Poursuivre avec les notations de la preuve du théorème \ref{decomph}. Remarquer que la fonctionnelle $\Phi_k:C(\sigma(A))\to\mathbb C$
\beq
f\mapsto\langle e_k,f(A)e_k\rangle
\eeq
est, pour $f$ positive, une fonctionnelle positive. D'après le théorème \ref{rk} et un résultat des exercices, il existe donc une mesure $\mu_k$ sur $\sigma(A)$ telle que 
\beq
\forall f\in C(\sigma(A)), \Phi_k(f)=\int_{\sigma(A)}fd\mu_k
\eeq
Si $f(A)e_k$, $g(A)e_k$ sont dans $H_k$, alors par le calcul fonctionnel et le théorème \ref{rk}, on a :
\beq
\langle f(A) e_k,g(A)e_k\rangle=\langle e_k,\overline{f(A)}g(A)e_k\rangle=\langle e_k,\overline{f}g (A)e_k\rangle=\int_{\sigma(A)}\overline fgd\mu_k
\eeq
Par construction de $H_k$, $x\in H_k$ si et seulement si il existe une suite $(f_n)_{n\in\mathbb N}\subset C(\sigma(A))$ telle que $\lim_{n}||x-f_n(A)e_k||=0$. La suite $(f_n(A))$ est donc une suite de Cauchy dans $H_k$ et on a 
\beq
||f_m(A)e_k-f_n(A)e_k||^2=\langle e_k,(\overline{f_m(A)-f_n(A)})(f_m(A)-f_n(A))e_k\rangle=||f_m-f_n||_{L^2(\sigma(A),\mu_k)}^2
\eeq
La suite converge donc également dans ${L^2(\sigma(A),\mu_k)}$ vers un élément $f_x\in{L^2(\sigma(A),\mu_k)}$. Manifestement, par continuité des normes $||f_x||_{L^2(\sigma(A),\mu_k)}^2=||x||_{H_k}^2$. L'application $H_k\to{L^2(\sigma(A),\mu_k)}, x\mapsto f_x$ est donc un isomorphisme unitaire. Puisque $H=\bigoplus H_k$, $x$ s'écrit de manière unique comme somme d'éléments $x_k$ de $H_k$. Par la discussion précédente, il existe pour chaque $k\in K$ un isomorphisme unitaire $V_k$ qui associe à $x_k$ une fonction $f_{x,k}\in L^2(\sigma(A),\mu_k)$. Poser :
\beq
V:H\to\bigoplus_{k\in K}L^2(\sigma(A),\mu_k)\hspace{0.5cm} x\mapsto\bigoplus_{k\in K}f_{x,k}
\eeq
%Remarquer alors que si $\oplus_{k\in K}x_k=x\in\mathcal H$ et si $\lim_n f_{n,k}(A)e_k=x_k$, alors par le calcul fonctionnel et la continuité de tous les opérateurs en question :
%\beq\begin{split}
%VAx=\osum_{k\in K}VAx_k=\oplus_{k\in K}VA\lim_n f_{n,k}(A)e_k=\oplus_{k\in K} V\lim_n (M_xf_{n,k})(A) e_k
%\end{split}\eeq
Montrer finalement que $VA=VM_x$. Par décomposition de $\mathcal H$ :
\beq
VAx=VA\bigoplus_{k\in K}x_k
\eeq
Par continuité de $A$ et de $V$ :
\beq
VAx=\bigoplus_{k\in K}VAx_k
\eeq
Par définition des $x_k$ et calcul fonctionnel :
\beq
VAx=\bigoplus_{k\in K}VA\lim_{n\to\infty}f_{n,k}(A)e_k
\eeq
Par continuité de $A$ :
\beq
VAx=\bigoplus_{k\in K}V\lim_{n\to\infty}M_xf_{n,k}(A)e_k
\eeq
Par continuité de $V$ :
\beq
VAx= \bigoplus_{k\in K} \lim_{n\to\infty}VM_xf_{n,k}(A)e_k
\eeq
Par définition de $V$ :
\beq
VAx= \bigoplus_{k\in K} \lim_{n\to\infty}M_xf_{n,k}
\eeq
Et finalement par continuité de $M_x$ et définitions de $V$ et $x$ :
\beq
VAx= \bigoplus_{k\in K} M_x\lim_{n\to\infty}f_{n,k}=M_xVx
\eeq
\qed\end{Pre}

\newpage
\part{Opérateurs Non-bornés}
\section{Notions fondamentales des opérateurs non-bornés}
\begin{Def}
Soit $(D(A),A)$ un opérateur. Le graphe $G(A)$ est défini par
\beq
G(A)=\{(x,y)\in\mathcal H\otimes\mathcal H:x\in D(A), y=Ax\}
\eeq
\end{Def}
\begin{Def} Si la fermeture du graphe d'un opérateur est le graphe d'un opérateur, on dit que cet opérateur est fermable.
\end{Def}
\begin{Def} L'opérateur $(D(B),B)$ est une extension de $(D(A),A)$ si et seulement si $G(A)\subset G(B)$. Noter $A\subset B$.
\end{Def}

\begin{Def}
Soient $(D(A_1),A_1)$ et $(D(A_2),A_2)$ deux opérateurs. Alors $D(A_1+A_2)=D(A_1)\cap D(A_2)$ et $D(A_2A_1)=\{x\in D(A_1):A_1x\in D(A_2)\}$ et on définit sur ces ensembles :
\beq
(A_1+A_2)x=A_1x+A_2x\hspace{0.5cm}(A_2A_1)x=A_2(A_1 x)
\eeq
\end{Def}

\begin{Def}
Soit $(D(A),A)$ un opérateur avec $D(A)$ dense dans $\mathcal H$. Le domaine de l'adjoint est défini par $D(A^*)$ est défini par :
\beq
D(A^*)=\{y\in\mathcal H:\exists!z_y\in\mathcal H:\forall x\in D(A), \langle y,Ax\rangle=\langle z_y,x\rangle\}
\eeq
Définir l'action de l'adjoint par $y\mapsto z_y$. 
\end{Def}
\begin{Def} Un opérateur densément défini est dit auto-adjoint si son graphe est égal à celui de son opérateur.
\end{Def}
\begin{Def} Un opérateur $A$ densément défini est dit normal si son domaine de définition est égal à celui de son adjoint et si pour chaque élément $x\in D(A)$, $||A^*x||=||Ax||$. 
\end{Def}

\begin{The}
Soit $(D(A),A)$ un opérateur sur $\mathcal H$. Alors :
\beq
G(A^*)=\left(G(-A)^{t}\right)^\perp
\eeq Avec le produit scalaire sur $\mathcal H\otimes\mathcal H$ définit comme la somme des produits scalaires des premières composantes avec les premières et des deuxièmes avec les deuxièmes. En particulier, l'adjoint d'un opérateur est toujours fermé.
\end{The}

\begin{Pre}
Soit $X=(x_1,x_2)\in \left (G(A)^t\right)^\perp$. De manière équivalente :
\beq\begin{split}
&\forall Y=(y_1,y_2)\in G(A)^{-t}: \langle x_1,y_2\rangle=\langle x_2,y_1\rangle\\
\Leftrightarrow&\forall Y=(y_1,y_2)\in G(A)^{-t}: \langle x_1,Ay_1\rangle=\langle x_2,y_1\rangle\\
\Leftrightarrow& x_2=A^*x_1
\end{split}\eeq

\qed\end{Pre}

\section{Les théorèmes de Banach-Steinhaus et du graphe fermé}

\begin{Lem}
Soit $A\in\mathcal L(X,Y)$ pour deux espaces normés $X$ et $Y$. Pour tout $x\in X$ et $r>0$, on a :
\beq
\sup\{||Ax'||:x'\in B_r(x)\}\geq||A||r
\eeq
\label{lembs}\end{Lem}
\begin{Pre}
Pour $\xi\in X$, 
\beq
\max\{||A(x-\xi)||,||A(x+\xi)||\}\geq\frac12\left(||A(x-\xi)||+||A(x+\xi)||\right)\geq||A\xi||
\eeq
Conclure en passant au supremum pour $\xi\in B_r(0)$.
\qed\end{Pre}

\begin{Def}
Une famille $\mathcal F\in\mathcal L(X,Y)$ est simplement bornée si pour tout $x\in X$, $\sup\{||Ax||:A\in\mathcal F\}<\infty$ ou uniformément bornée sur $\sup\{||A||\in\mathcal F\}<\infty$.
\end{Def}

\begin{The}\textbf{Théorème de Banach-Steinhaus}
Soient $X$ un espace de Banach, $Y$ un espace normé et $\mathcal F\subseteq\mathcal L(X,Y)$ une famille simplement bornée, alors $\mathcal F$ est uniformément bornée.
\label{bs}\end{The}

\begin{Pre}
Supposer que la famille ne soit pas uniformément bornée. On peut alors choisir une suite d'opérateur $(A_n)\subseteq F$ telle que $||A_n||\geq4^n$. Poser $x_0=0$ et $x_n\in B_{3^{-n}}(x_{n-1})$ et $||A_nx_n||\geq\frac23||A_n||3^{-n}$, qui existe par le lemme précédent \ref{lembs}. La suite $(x_n)$ étant de Cauchy, elle converge vers $x\in X$. De plus, $||x-x_n||\leq\frac{3^{-n}}2$. Mais, par construction des $x_n$ et l'inégalité triangulaire inverse, on trouve :
\beq
||A_nx||\geq\frac{3^{-n}}6||A_n||\geq\frac{4^n}{6\times3^n}
\eeq 
ce qui contredit la majoration simple.
\qed\end{Pre}

\begin{Def} Soit $\mathcal H$ un espace de Hilbert. On dit qu'un sous-ensemble $D$ est faiblement séquentiellement compact si pour toute suite $(x_n)$ dans $ D$, il existe $x\in D$ et une sous-suite $(y_m)\subseteq(x_n)$, telle que pour tout $v\in \mathcal H$ :
\beq
\lim_{n\to\infty}\langle y_n,v\rangle=\langle x,v\rangle 
\eeq
\end{Def}

\begin{The}\textbf{Théorème de Bolzano-Weierstrass} Le disque unité $\overline {B_1(0)}$, dans un espace de Hilbert $\mathcal H$ séparable, est séquentiellement compact.
\label{bolzano}\end{The}

\begin{Pre}
Soient $(x_n)\subset \overline{B_1(0)}$ et $(e_n)$ un système ortho-normé et complet de $\mathcal H$. Alors, pour tout $m\in\mathbb N$, la suite $(\langle x_n,e_m)_{n\in\mathbb N}$ est une suite bornée de $\mathbb C$. Pour $m=1$, on peut choisir une sous-suite $(x_{n_k})_{k\in\mathbb N}$ qui converge vers un élément de l'espace de Hilbert dénoté par $z_1$. Poser $y_1$, d'indice $N_1,$ la premier élément de la sous-suite tel que pour tous les $k\geq N_1$, $|\langle x_{n_k},e_1\rangle|<1$. Répéter le processus pour chaque $m$ de manière à avoir une distance plus petite que $1/2^{m-1}$. Par itération, on construit une sous-suite $(y_m)$ de $(x_n)$ et une suite $(z_n)\subset \mathbb C$ telles que :
\beq
\forall k\in\mathbb N:\lim_{m\to\infty}\langle y_m,e_k\rangle=z_k
\eeq
Pour tout $v\in\text{Vect}\left(e_i\right)$, on a alors $\lim_{m\to\infty}\langle y_m,v\rangle$ existe et :
\beq
\left|\lim_{m\to\infty}\langle y_m,v\rangle\right|\leq\limsup_{m\to\infty}|\langle y_m,v\rangle|\leq||v||
\eeq
L'application $\text{Vect}\left(e_i\right)\to\mathbb C$, $v\mapsto\lim_{m\to\infty}\langle y_m,v\rangle$ est alors une fonctionnelle linéaire bornée. Puisque $(e_i)$ est une famille dense de l'espace de Hilbert, on peut étendre l'opérateur à tout l'espace. Le théorème de Riesz-Fréchet permet alors de conclure.

\qed\end{Pre}

\begin{Lem}
Soit $(D(T),T)$ un opérateur fermé et densément défini. Alors $D(T^*)$ est dense.
\label{densitedomainead}\end{Lem}
\begin{Lem} Soit $(D(T),T)$ un opérateur fermé sur un espace de Hilbert séparable $\mathcal H$ avec $D(T)=\mathcal H$. Alors $D(T^*)=\mathcal H$.
\end{Lem}

\begin{Pre}
Comme $D(T)$ est dense, $T^*$ existe. Par le lemme \ref{densitedomainead}, $D(T^*)$ est dense. Soit $v\in\mathcal H$ et choisir une suite $(v_n)\subset D(T^*)$, bornée, telle que $\lim_{n\to\infty}v_n=v$.
Pour $w\in\mathcal H$, on a $\langle v_n,Tw\rangle=\langle T^*v_n,w\rangle$ et 
\beq
\sup_{n\in \mathbb N}| \langle T^*v_n,w\rangle|\leq\sup_{n\in \mathbb N}||Tw||||v_n||<\infty
\eeq
Par le théorème de Banach-Steinhaus \ref{bs}, $\sup_{n\in\mathbb N}<\infty$. Par Bolzano-Weierstrass \ref{bolzano}, il existe une sous-suite $\left(T^*v_{\sigma(n)}\right)_{n\in\mathbb N}$ de $\left(T^*v_n\right)$ et un $y\in\mathcal H$, tels que
\beq
\forall x\in\mathcal H:\lim_{n\to\infty}\langle T^*v_{\sigma(n)},x\rangle=\langle y,x\rangle=\langle v,Tx\rangle
\eeq
\qed\end{Pre}

\begin{The} \textbf{Théorème du graphe fermé} Soit $(D(T),T)$ un opérateur fermé sur un espace de Hilbert $\mathcal H$ séparable avec $D(T)=\mathcal H$. Alors $T$ est borné. \label{closedgraph}\end{The}
\label{tgf}\begin{Pre}
Supposer $T$ non-borné. Il existe une suite $(u_n)\subset B_1(0)$, telle que $\lim_{n\to\infty}||Tu_n||=\infty$. D'un autre côté, pour $x\in\mathcal H=D(T^*)$,
\beq
\sup_{n\in\mathbb N}|\langle Tu_n,x\rangle|=
\sup_{n\in\mathbb N}|\langle u_n,T^*x\rangle|\leq\sup_{n\in\mathbb N}||u_n||||T^*x||=||T^*x||
\eeq
\qed\end{Pre}

\section{Le coeur et l'adjoint essentiel}

\begin{Def} Pour un opérateur $(D(T),T)$ fermé, définir l'ensemble résolvant $\rho(T)$ par :
\beq
\rho(T)=\{z\in\mathbb C:\text{ker}(T-z)={0}\text{ et }\mathcal H=\text{ran}(T-z)\}
\eeq
et le spectre $\sigma(T)=\mathbb C\backslash \rho(T)$. Pour $z\in\rho(T)$, $R(z,T)=(T-z)^{-1}$ est appelé la résolvante.
\end{Def}

\begin{Prop}
Une définition équivalente de l'ensemble résolvant est la suivante :
\beq
\rho(T)=\{\lambda\in\mathbb C:\exists A\in\mathcal L(\mathcal H):A(T-\lambda)= 1_{D(T)}\text{ et }(T-\lambda)A= 1\}
\eeq
\end{Prop}

\begin{Pre}
Montrer que la résolvante est bornée. Puisque $(D(T),T)$ est fermé, le graphe de $T$ l'est. Sa transposée l'est donc aussi, tout comme $G(T-\lambda)$, pour $\lambda\in\rho(T)$, ou encore $G(T-\lambda)^t=G(R(\lambda,t))$. L'opérateur $R(\lambda,T)$ est donc fermé, défini sur tout l'espace de Hilbert et le théorème du graphe fermé \ref{tgf} implique alors que $R(\lambda,T)\in\mathcal L(\mathcal H)$. 
\qed\end{Pre}

\begin{Prop} Soit $(D(T),T)$ un opérateur fermé. Alors $z\in \rho(T)$ si et seulement si $\overline z\in\rho(T^*)$ et 
\beq
R(z,T)^*=R(\overline z,T^*)
\eeq
\end{Prop}

\begin{Pre}
Soit $z\in\mathbb C$. Pour que $(T-z)x$ soit bien défini, il faut et suffit que $Tx$ le soit, donc $x\in D(T)$. Donc $D(T-z)=D(T)$. Puis, pour chaque $x\in D(T)$, et $y\in\mathcal H$, dire que $x\mapsto \langle y,(T-z)x\rangle$ est continu revient à dire que $x\mapsto \langle y,Tx\rangle-\langle \overline zy,x\rangle$ est continu, c'est-à-dire $x\in D(T^*)$. Ainsi, $D(T-z)^*=D(T^*)$ et $(T-z)^*=T^*-\overline z$. Soit $z\in \rho(T)$. Cela implique donc que $R(z,T)\in\mathcal L(\mathcal H)$ avec 
\beq
(T-\lambda)R(z,T)=1_{\mathcal H}
\eeq
et 
\beq
R(z,T)(T-z)=1_{D(T)}
\eeq
Ainsi, $R(z,T)^*\in\mathcal L(\mathcal H)$ et 
\beq\begin{split}
&G(R(z,T)^*)={G(-R(z,T))^t}^\perp={{G(-(T-z))^\perp=G(-(T-z))^t}^t}^\perp\\&={{G(-(T-z))^t}^\perp}^t=G((T-z)^*)^t=G(T^*-\overline z)^t
\end{split}\eeq
Ceci montre donc que $R(z,T)^*\in\mathcal L(\mathcal H)$ est l'inverse borné de $T^*-\overline z$ et que par conséquence, $\overline z\in\rho(T^*)$, ainsi que $R(z,T)^*=R(\overline z,T^*)$. La réciproque est une conséquence de cet argument, puisque $T$ étant fermé, on a $T={T^*}^*$.
\qed\end{Pre}

\begin{Def} Un opérateur $(D(T),T)$ est dit symétrique si $T\subset T^*$ et $D(T)$ est dense.
\end{Def}

\begin{Lem}
Un opérateur $(D(T),T)$ symétrique est fermable. De plus, ker$(T\pm i)=\{0\}$ et 
\beq
\overline{\text{Ran}(T\pm i)}=\text{Ran}(\overline T\pm i)
\eeq
où $\overline T={T^*}^*$ est la fermeture de $T$.
\label{tpmi}\end{Lem}
\begin{Pre}
Comme l'adjoint d'un opérateur densément défini existe toujours et qu'il est fermé, on a pour un opérateur symétrique que $G(T)$ est contenu dans le graphe $G(T^*)$, qui lui est fermé. On a alors que $T$ est fermable et que $\overline{G(T)}={G(T)^\perp}^\perp={G(-T^*)^t}^\perp=G({T^*}^*)$. \\
Puisque $T\subset T^*$, on a pour $x\in D(T)$, $\langle x,Tx\rangle=\langle T^*x,x\rangle=\langle Tx,T\rangle=\overline{\langle x,Tx\rangle}$. Ainsi
\beq\begin{split}
||(T\pm i)x||^2&=\langle Tx,Tx\rangle\mp i \langle x,Tx\rangle\pm i\langle Tx,x\rangle +\langle x,x\rangle\\
&=||Tx||^2+||x||^2\geq||x||^2,||Tx||^2
\end{split}\eeq
Ceci implique que Ker$(T\pm i)=\{0\}$. De plus, ceci montre que $(y_n)=\left((T \pm i)x_n\right)$ est une suite de Cauchy dans l'image de $T\pm i$ si et seulement $(x_n)$ est de Cauchy dans $D(T)$ et $(Tx_n)$ est de Cauchy dans l'image de $T$, c'est-à-dire si et seulement si $(\lim x_n,\lim y_n)\in G(\overline T)$. Donc, $\overline {\text{Ran}(T\pm i)}=\text{Ran}(\overline T\pm i)$.
\qed\end{Pre}

\begin{The}\textbf{Théorème de von Neumann} Soit $(D(T),T)$ un opérateur symétrique et fermé sur un espace de Hilbert séparable $\mathcal (\mathcal H)$. Alors $T$ est auto-adjoint si et seulement si $i\in \rho(T)$. Dans un tel cas $\sigma(T)\subseteq\mathbb R$. \label{tvn}\end{The}

\begin{Pre}
Supposons d'abord que $\pm i\in \rho(T)$. Alors, $R(\pm i,T)\in \mathcal L(\mathcal H)$ existent et Ran$(T\pm i)=\mathcal H$. Pour $x\in D(T^*)$, il existe alors un $y\in D(T)$, tel que $(T^*\pm i)x=(T\pm i)y$. Puisque $T$ est symétrique, $T\subset T^*$. Mais par la proposition précédente, $\pm i\in \rho(T^*)$ aussi, et $x=y$. Ainsi, $D(T)=D(T^*)$ et $T=T^*$.\\
Supposons maintenant que $T=T^*$. Alors, $T$ est à fortiori symétrique, $T=\overline T$ et par le lemme précédent \ref{tpmi}, Ker$(T\pm i)=\{0\}$ et l'image de $t\pm i$ est fermée. Si $x$ est orthogonal à l'image de $T\pm i$, alors $\forall y\in D(T)$ : $\langle x, (T\pm i)y\rangle =0$, de sorte que $D(T^*)=D(T)$ et $\langle (T\mp i)x, y\rangle =0$. Comme $D(T)$ est dense dans $\mathcal H$, ceci implique que $(T\mp i)x=0$, donc $x\in \text{Ker}(T\pm i)$, donc $x=0$. Ainsi, l'image de $T\pm i$ est l'espace de Hilbert entier. Comme $T\pm i$ sont des opérateurs fermés sur $D(T)$, on a que $(T\pm i)^{-1}$ sont fermés aussi et définis sur tout $\mathcal H$. Par le théorème du graphe fermé, $(T\pm i)^{-1}\in\mathcal L(\mathcal H)$ et $\pm i\in \rho(T)$.\\
Supposons enfin que $T=T^*$ et que $z=a+ib$, $a\in\mathbb R$, $b\in\mathbb R\backslash{0}$. Alors $b^{-1}(T-a)$ est auto-adjoint aussi et $\pm i\in \rho(b^{-1}(T-a))$, d'où on conclut que $\pm ib\in\rho(T-a)$, où encore, que $a\pm ib\in \rho(T)$. Conclure que $\sigma(T)\subseteq \mathbb R$.
\qed\end{Pre}

\begin{Def}
Un opérateur $(D(T),T)$ symétrique est dit essentiellement auto-adjoint si sa fermeture $\overline T={T^*}^*$ est auto-adjointe. Dans ce cas, $D(T)$ est un coeur pour ${T^*}^*$.
\end{Def}

\begin{Cor} Un opérateur $(D(T),T)$ symétrique est essentiellement auto-adjoint si et seulement si l'une des deux conditions suivantes est vérifiée :
\begin{itemize}
\item $\overline{\text{Ran}(T\pm i)}=\mathcal H$
\item $\text{Ker}(T^*\pm i)=\{0\}$
\end{itemize}
\end{Cor}

\begin{Pre} Commençons par noter que ces deux conditions sont équivalentes. En effet :
\beq
\overline{\text{Ran}(T\pm i)}=\mathcal H\Leftrightarrow \text{Ran}(T\pm i)^\perp=\{0\}
\eeq
Mais l'orthogonal de l'image d'un opérateur n'est rien d'autre que le noyau de son adjoint et $(T\pm i)^*=T^*\mp i$.\\
Par le lemme précédent, $\overline{\text{Ran}(T\pm i)}=\text{Ran}(\overline T\pm i)$, donc $\overline{\text{Ran}(T\pm i)}=\mathcal H$ est équivalent à dire que $\overline T\pm i$ est une bijection entre $D(\overline T)$ et $\mathcal H$.\\
Puisque $\overline T\pm i$ est fermé, $(\overline T\pm i)^{-1}$ l'est aussi et, d'après le théorème du graphe fermé \ref{tgf}, on a que $(\overline T\pm i)^{-1}\in\mathcal L(\mathcal H)$. On a alors $\pm i\in \rho(\overline T)$, ce qui, par le théorème de von Neumann \ref{tvn}, est équivalent à dire que $\overline T$ est auto-adjoint.
\qed\end{Pre}

\section{Décomposition spectrale d'opérateurs non-bornés}

\begin{Prop}
Pour un opérateur $A$ densément défini et fermé, on a la décomposition :
\beq
\mathcal H\bigoplus\mathcal H=G(A^*)\bigoplus G(-A)^{t}
\eeq
\label{decotri}\end{Prop}
\begin{Pre} Clair puisque $G(A^*)$ est fermé, puisque qu'écrit comme le complément orthogonal d'un sous-espace fermé et convexe.
\qed\end{Pre}

\begin{Lem}
Soient $(D(A),A)$ un opérateur densément définit et fermé, et $u,v\in\mathcal H$. Il existe alors un unique couple $(x,y)\in D(A)\times D(A^*)$ tel que $y-Ax=u$ et $x+A^*y=w$. De plus :
\beq
||u||^2+||w||^2=||x||^2+||y||^2+||Ax||^2+||A^*y||^2
\eeq
\label{identite}
\end{Lem}

\begin{Pre}
Puisque $\mathcal H\bigoplus\mathcal H=G(A^*)\bigoplus G(A)^{-t}$, voir proposition \ref{decotri}, et que $(u,v)\in \mathcal H\bigoplus\mathcal H$, il existe une unique décomposition :
\beq
(u,v)=(y,A^*y)+(-Ax,x)
\eeq
avec $(x,y)\in D(A)\times D(A^*)$. Par le théorème de Pythagore :
\beq
||u||^2+||w||^2=||(y,A^*y)+(-Ax,x)||_{\mathcal H\bigoplus\mathcal H}^2=||y||^2+||A^*y||^2+||Ax||^2+||x||^2
\eeq
\qed\end{Pre}

\begin{The} Soit $(D(A),A)$ un opérateur densément défini et fermé. Alors :
\begin{enumerate}
\item $(D(A^*A),1+A^*A)$ est un opérateur auto-adjoint
\item $1+A^*A$ est une bijection entre $D(A^*A)$ et $\mathcal H$
\item $(1+A^*A)^{-1}\in\mathcal L(\mathcal H)$ est auto-adjoint et $||(1+A^*A)^{-1}||\leq1$
\item $D(A^*A)$ est un coeur pour $A$
\end{enumerate}
\end{The}

\begin{Pre}
Commençons par prouver que $1+A^*A$ est une bijection entre $D(A^*A$ et $\mathcal H$. En appliquant le lemme \ref{identite} aux cas $v=0$ et $w\in\mathcal H$, on trouve des couples $(x,y)\in D(A)\times D(A^*)$ tels que $0=y-Ax$ et $w=x+A^*y$. Ceci implique que pour tout $w\in\mathcal H$, il existe un unique $x\in D(A)$, tel que $Ax\in D(A^*)$ et $1+A^*A)x=w$. L'opérateur établit alors bien une bijection entre $D(A^*A$ et $\Hcal$. \\
$(1+A^*A)^{-1}\in\mathcal L(\mathcal H)$ et $||(1+A^*A)^{-1}||\leq1$ suivent de $||w||^2=||x||^2+2||Ax||^2+||A^*Ax||^2\geq||x||^2$.\\
Montrons maintenant que $(D(A^*A),1+A^*A)$ est auto-adjoint. Tout d'abord, cet opérateur est fermé car le graphe d'un opérateur borné est toujours fermé, que $G(B^{-1})=G(B)^t$ et que le graphe d'un opérateur est fermé si et seulement si le graphe transposé l'est. Remarquer que $D(A^*A)$, et donc $D(1+A^*A)$, est dense dans $\Hcal$. Soit $y\in D(A^*A)^\perp$ et montrons que $y=0$. Il doit exister un unique $x\in D(1+A^*A)$ tel que $y=x+A^*Ax$. Alors :
\beq
0=\vev{x,y}=\vev{x,x+A^*Ax}=||x||^2+||Ax||^2
\eeq
donc $x=0=Ax=y$. Soit désormais $y\in D(1+A^*A)$. Il existe alors un unique $z\in\Hcal$ tel que :
\beq
\vev{y,x+A^*Ax}=\vev{z,x}
\eeq
Mais $1+A^*A$ étant une bijection, il doit exister un unique $y'\in D(1+A^*A)$ tel que $z=y'+A^*Ay'$. Mais alors pour tout $x\in D(1+A^*A)$ :
\beq
\vev{y,x+A^*Ax}=\vev{z,x}=\vev{y',x+A^*Ax}
\eeq
de sorte que $y=y'$ et $z=y+A^*Ay$. Prenons maintenant $x,y\in\Hcal$ arbitraires. Par bijectivité et le fait que $1+A^*A$ est auto-adjoint sur $D(A^*A)$ :
\beq\begin{split}
\vev{y,(1+A^*A)^{-1}x}=\vev{(1+A^*A)^{-1}y,x}
\end{split}\eeq
Montrer enfin le dernier point. Supposer $(x,Ax)\in G(A)$ est orthogonal dans $\Hcal\bigoplus\Hcal$ au graphe de $A$ restreint à $D(A^*A)$ et montrons que $x=0$. Cela provient du fait que pour tout $y\in D(A^*A)$, on a que :
\beq
0=\vev{(x,Ax),(y,Ay)}=\vev{x,y}+\vev{x,(1+A^*A)y}
\eeq
\qed\end{Pre}

\begin{The}
Soit $(D(A),A)$ un opérateur normal. Alors
\begin{enumerate}
\item $x\in D(A^2)\implies (1+A^*A)^{-1/2}x\in D(A^2)$
\item Pour $x\in D(A)$, $A(1+A^*A)^{-1}x=(1+A^*A)^{-1}Ax$ et $A^*(1+A^*A)^{-1}x=(1+A^*A)^{-1}A^*x$
\item Pour $x\in D(A^2)$, $A(1+A^*A)^{-1/2}x=(1+A^*A)^{-1/2}Ax$ et $A^*(1+A^*A)^{-1/2}x=(1+A^*A)^{-1/2}A^*x$
\item $(1+A^*A)^{-1/2}A$ et $(1+A^*A)^{-1/2}A^*$ sont bornés sur $D(A^2)$, de norme inférieure à 1, de sorte qu'il existe des extensions bornées uniques $T_A$ et $T_{A^*}$. De plus, $T_A^*=T_{A^*}$
\item $1-T_{A^*}T_A=(1+A^*A)^{-1}$
\end{enumerate}
\label{avantder}\end{The}

\begin{Pre}
Commencer par la discussion suivante. On a $D(A^2)=D(A^*A)=D(AA^*)=D({A^*}^2)$. Par exemple :
\beq\begin{split}
x\in D(A^*A)&\Leftrightarrow \forall y\in D(A) : y\mapsto\vev{y,A^*Ax}\in\mathcal L(D(A),\mathbb C)\\
&\Leftrightarrow \forall y\in D(A) : y\mapsto\vev{Ay,Ax}\in\mathcal L(D(A),\mathbb C)\\
&\Leftrightarrow \forall y\in D(A) : y\mapsto \sum_{k=0}^3i^k\vev{A(y+i^kx),A(y+i^kx)}\in\mathcal L(D(A),\mathbb C)\\
&\Leftrightarrow \forall y\in D(A) : y\mapsto \sum_{k=0}^3i^k||A(y+i^kx)||^2\in\mathcal L(D(A),\mathbb C)\\
&\Leftrightarrow \forall y\in D(A) : y\mapsto \sum_{k=0}^3i^k||A^*(y+i^kx)||^2\in\mathcal L(D(A),\mathbb C)\\
&\Leftrightarrow \forall y\in D(A) : y\mapsto\vev{A^*y,A^*x}\in\mathcal L(D(A),\mathbb C)\\
&\Leftrightarrow A^*x\in D(A)\\
&\Leftrightarrow x\in D(AA^*)
\end{split}\eeq
De plus, si $x\in D(A^2)$, alors :
\beq
\vev{x,A^*Ax}=\vev{Ax,Ax}=||Ax||^2=||A^*x||^2=\vev{Ax,Ax}=\vev{x,AA^*x}
\eeq
et en appliquant encore une fois l'identité de polarisation, on a que pour un tel opérateur normal $(D(A),A)$ :
\beq
\forall x\in D(A^*A)=D(AA^*) : AA^*x=A^*Ax
\eeq
Puisque $(1+A^*A)^{-1}$ est borné, on peut par le calcul fonctionnel sur les opérateurs bornés définir l'opérateur $(1+A^*A)^{-1/2}$. Celui-ci sera aussi borné, auto-adjoint et de norme inférieure à 1. Manifestement :
\beq
(1+A^*A)^{-1/2}(1+A^*A)^{-1/2}=(1+A^*A)^{-1}
\eeq
Prouvons désormais chaque point explicitement.
\begin{enumerate}
\item Si $x\in D(A^2)$, il doit exister $y\in\Hcal$ tel que $x=(1+A^*A)y$. Mais alors, par le calcul fonctionnel, on a $(1+A^*A)^{-1/2}x=(1+A^*A)^{-3/2}y=(1+A^*A)^{-1}(1+A^*A)^{-1/2}y\in D(A^2)$.
\item Si $x\in D(A)$, il existe $y\in D(A^2)$ tel que $(1+A^*A)y=x$. Ceci montre que $A^*Ay=x-y\in D(A)$, d'où $y\in D(AA^*A)$, ou encore, $Ay\in D(AA^*)=D(A^*A)$. Ainsi, on a :
\beq\begin{split}
&A(1+A^*A)^{-1}x=Ay=(1+A^*A)^{-1}(1+A^*A)Ay=(1+A^*A)^{-1}(Ay+A^*AAy)\\
&=(1+A^*A)^{-1}(Ay+AA^*Ay)=(1+A^*A)^{-1}A(1+A^*A)y=(1+A^*A)^{-1}Ax
\end{split}\eeq 
Le raisonnement pour l'autre égalité est analogue.
\item Par définition du calcul fonctionnel, il doit exister une suite $(p_n)$ convergeant uniformément sur $\sigma(((1+A^*A)^{-1}))$ vers la fonction $x\mapsto \sqrt x$ de telle sorte que $||(1+A^*A)^{-1/2}-p_n((1+A^*A)^{-1})||\to0$. Pour $x,y\in D(A^2)$ : 
\beq\begin{split}
\vev{x,A(1+A^*A)^{-1/2}y}&=\vev{A^*x,(1+A^*A)^{-1/2}y}=\vev{A^*x,\lim_{n\to\infty} p_n((1+A^*A)^{-1})y} = \lim_{n\to\infty} \vev{A^*x,p_n((1+A^*A)^{-1})y}\\
&=\lim_{n\to\infty} \vev{x,Ap_n((1+A^*A)^{-1})y}= \lim_{n\to\infty}\vev{x,p_n((1+A^*A)^{-1})Ay}\\
& = \vev{x,\lim_{n\to\infty} p_n((A+A^*A)^{-1})Ay}=\vev{x,(1+A^*A)^{-1/2}Ay} 
\end{split}\eeq
Les éléments de matrices de $(1+A^*A)^{-1/2}A$ et de $A(1+A^*A)^{-1/2}$ étant égaux sur $D(A^2)$ qui est dense dans $\Hcal$, on peut conclure.
\item Il existe des extensions bornées de $T_A$ et $T_{A^*}$ de $A(1+A^*A)^{-1/2}$ et $A^*(1+A^*A)^{-1/2}$ respectivement. De plus, $T^*_A =T_{A^*}$ et $||T_A||\leq 1$. Pour $x\in D(A^2)$, on a que :
\beq\begin{split}
&||(1+A^*A)^{-1/2}Ax||^2=\vev{Ax,(1+A^*A)^{-1/2}(1+A^*A)^{-1/2}Ax}=\vev{Ax,(1+A^*A)^{-1}Ax}=\vev{x,A^*(1+A^*A)^{-1}Ax}\\
&=\vev{x,A^*A(1+A^*A)^{-1}x}\leq\vev{x,A^*A(1+A^*A)^{-1}x}+\vev{x,(1+A^*A)^{-1}x}=\vev{x,(1+A^*A)(1+A^*A)^{-1}}=||x||^2
\end{split}\eeq L'opérateur $(1+A^*A)^{-1/2}A$ est donc bien borné sur $D(A^2)$, de norme inférieure à 1, et possède donc une unique extension sur tout $\Hcal$. Il en va de manière similaire pour $(1+A^*A)^{-1/2}A^*$. Finalement, pour $x,y\in D(A^2)$, on a :
\beq
\vev{x,A(1+A^*A)^{-1/2}y}=\vev{A^*x,(1+A^*A)^{-1/2}y}=\vev{(1+A^*A)^{-1/2}A^*x,y}
\eeq
ce qui montre bien que $(A(1+A^*A)^{-1/2})^*=(1+A^*A)^{-1/2}A^*$. L'extension de l'adjoint étant l'adjoint de l'extension, $T_A^*=T_{A^*}$.
\item Prendre à nouveau $x,y\in D(A)$ :
\beq
\begin{split}
\vev{x,(1-T_{A^*}T_A)y}&=\vev{x,y}-\vev{T_Ax,T_Ay}=\vev{x,y}-\vev{(1+A^*A)^{-1/2}Ax,(1+A^*A)^{-1/2}Ay}\\
&=\vev{x,y}-\vev{Ax,(1+A^*A)^{-1/2}(1+A^*A)^{-1/2}Ay}=\vev{x,y}-\vev{Ax,(1+A^*A)^{-1}Ay}\\
&=\vev{x,y}-\vev{x,A^*A(1+A^*A)^{-1}y}=\vev{x,(1+A^*A)(1+A^*A)^{-1}y}-\vev{x,A^*A(1+A^*A)^{-1}y}\\
&=\vev{x,(1+A^*A)^{-1}y}
\end{split}
\eeq Ces deux opérateurs bornés ont donc les mêmes éléments de matrice sur un ensemble dense de $\Hcal$ et sont donc identiques.
\end{enumerate}
\qed\end{Pre}

\begin{The}
Soit $(D(A),A)$ un opérateur auto-adjoint. Alors, il existe une famille dénombrable $(\nu_n)_{n\in B}$ de mesures boreliennes, régulières et finies sur $\sigma(A)$ et un unique opérateur unitaire $U:\mathcal H\to\bigoplus_{n\in B}L^2(\sigma(A),\nu_n)$ tels que
\begin{enumerate}
\item $UD(A)=\{(f_n)_{n\in B}\in\bigoplus_{n\in B}L^2(\sigma(A),\nu_n):(xf_n(x))_{n\in B}\in\bigoplus_{n\in B}L^2(\sigma(A),\nu_n)\}$
\item Sur $D(A)$, on a $A=U^{-1}M_xU$
\end{enumerate}
\end{The}

\begin{Pre}
Commençons par la discussion suivante. Si $(D(A),A)$ est auto-adjoint, il est à fortiori normal. Par le théorème précédent \ref{avantder}, $T_A$ est donc un opérateur auto-adjoint borné de norme inférieure à 1. Par conséquent son spectre est dans $[-1,1]$ et par le théorème de la décomposition spectrale pour opérateurs auto-adjoints bornés \ref{decompoborne}, il existe une famille de mesures régulières boréliennes $(\mu_n)_{n\in B}$ sur $\sigma(A)$ et avec $|B|\leq|\mathbb N|$ et un opérateur unitaire $V$ :
\beq
V:\Hcal\to\bigoplus_{n\in B}L^2(\sigma(A),\mu_n) :VT_AV^{-1}=M_x
\eeq
Par le théorème \ref{avantder}, on a alors que :
\beq
(1+A^*A)^{-1}=1-T_A^2=V^{-1}(1-M_x^2)V
\eeq
Clairement :
\beq
VD(A^2)=V\text{Ran}(1+A^*A)^{-1}=\text{Ran}(1-M_x^2)=\left\{((1-x^2)f_n(x))_{n\in B}:(f_n)_{n\in B}\in\bigoplus_{n\in B}L^2(\sigma(A),\mu_n)\right\}
\eeq
Manifestement, on a aussi $V(1+A^*A)^{-1/2}V^{-1}=M_{\sqrt{1-x^2}}$. Puis sur $VD(A^2)$, on a :
\beq
M_x=VA(1+A^2)^{-1/2}V^{-1}=V(1+A^2)^{-1/2}AV^{-1}=V(1+A^2)^{-1/2}V^{-1}VAV^{-1}=M_{\sqrt{1_x^2}}VAV^{-1}
\eeq
de sorte que sur $VD(A^2)$, $VAV^{-1}=M_{\frac{x}{\sqrt{1-x^2}}}$. Puisque $D(A^2)$ est un coeur pour $A$, $G(A)$ sera la fermeture dans $\Hcal\bigoplus\Hcal$ du graphe de $A$ restreint à $D(A^2)$. Puisque $V$ est unitaire, on aura que $(D(A),A)$ sera unitairement équivalent à la fermeture du graphe $M_{\frac{x}{\sqrt{1_x^2}}}$ restreint à $VD(A^2)$. De même, (D(A),A) est unitairement équivalent $(VD(A),M_{\frac{x}{\sqrt{1-x^2}}})$. Par le point $5$ du théorème \ref{avantder}, $(1+A^2)^{-1}=1-T_A^2$ est une injection, de sorte que par unitarité de $V$, l'opérateur $1-M_x^2$ est une bijection dans $\bigoplus_{n\in B}L^2(\sigma(A),\mu_n)$. Il en résulte que pour tout $n\in B$, $\mu_n\{-1,1\}=0$ car sinon le vecteur $\bigoplus_{k\in B} x_k$ avec $x_k=\delta_{nk}\chi_{\{-1,1\}}$ serait un vecteur non-nul avec $1-M_x^2\bigoplus_{k\in B} x_k=0$. Considérons alors $$\varphi:\mathbb R\to]-1,1[, x\mapsto y=\varphi(x)=\frac x{\sqrt{x^2+1}}$$
Cette fonction est clairement une bijection continue avec inverse $\varphi^{-1}(y)=\frac y{\sqrt{1-x^2}}$. Si $\Sigma_n\subseteq\mathcal P(]-1,1[)$ est la $\sigma$-algèbre borélienne sur laquelle est définie $\mu_n$, alors :
\beq
\Sigma_n^{\varphi}=\{\varphi^{-1}\{E\}:E\in \Sigma_n\}
\eeq
est une $\sigma$-algèbre borélienne de $\mathcal P(\mathbb R)$. On définit alors une mesure $\nu _n$ :
\beq
\nu_n:F=\varphi^{-1}\{E\}\in \Sigma_n^\varphi\to\mathbb R_+, F\mapsto \mu_n(E)
\eeq 
Il est alors clair que $f$ est une fonction $\Sigma_n$-mesurable si et seulement si $f\circ\varphi$ est $\Sigma_n^\varphi$-mesurable et que $s$ est une fonction $\Sigma $-simple si et seulement si $s\circ\varphi$ est une fonction $\Sigma_n^\varphi$-simple. Par définition de $\nu_n$, il est alors clair que pour une fonction $\Sigma_n^\varphi$-simple :
\beq
\int_{\mathbb R}sd\nu_n=\int_{\sigma(A)}s\circ\varphi^{-1}d\mu_n
\eeq
et que par conséquent, la composition des fonctions par $\varphi$ induit un isomorphisme unitaire $W_n:L^2(\sigma(A),\mu_n)\to L^2(\mathbb R,\nu_n)$. $\varphi$ étant une bijection continue entre intervalles réels de sorte que des compacts de $\mathbb R$ sont envoyés sur des compacts de $]-1,1[$, et que la régularité de $\mu_n$ a alors comme conséquence la régularité de $\nu_n$. La composition par $\varphi$ montre alors aussi que
\beq
W_nM_{\frac{y}{\sqrt{1-y^2}}}=M_xW_n
\eeq
 sur l'ensemble $VD(A)_n$ et la composition par $\varphi$ nous donne que 
 \beq
 W_nVD(A)_n=\left\{\frac{f(x)}{\sqrt{1+x^2}}:f\in L^2(\mathbb R,\nu_n)\right\}
 \eeq
Mais alors, 
\beq
\begin{split}
g\in W_nVD(A)_n\Leftrightarrow\int_{\mathbb R}\overline {g(x)}(1+x^2)g(x)d\nu_n<\infty\Leftrightarrow g,xg(x)\in L^2(\mathbb R,\nu_n)
\end{split}
\eeq 
Ainsi :
\beq
W_nVD(A)_n=\left\{f\in L^2(\mathbb R,\nu_n):M_xf\in L^2(\mathbb R,\nu_n)\right\}
\eeq
En définissant $W=\bigoplus_{n\in B}W_n$ et $U = WV$, on la conclusion du théorème.
\qed\end{Pre}










\end{document}

