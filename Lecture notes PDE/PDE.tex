\documentclass[a4paper,10pt,preprint,aps,tightenlines,showpacs,superscriptaddress]{revtex4}	%Rajouter a4paper,12pt suites à relecture des instructions
\usepackage[top=2cm, bottom=2cm, left=1cm, right=1cm]{geometry}	% new package
\usepackage[dvips]{graphicx}
\usepackage{dcolumn}
\usepackage{epsfig}
\usepackage{color}
\usepackage{bm}
\usepackage{gensymb}
\usepackage{amsmath}
\usepackage{amsthm}
\usepackage{appendix}
\usepackage[english]{babel}
\usepackage[T1]{fontenc}
\usepackage{fancyhdr}
\usepackage{amsfonts}
\usepackage{comment}

\def\bra#1{\mathinner{\langle{#1}|}}
\def\ket#1{\mathinner{|{#1}\rangle}}
\def\braket#1{\mathinner{\langle{#1}\rangle}}

\graphicspath{{tpillustrations/}}

\def\slash#1{\not\!#1}
\def\slashb#1{\not\!\!#1}
\def\delsla{\not\!\partial}

\newcommand{\red}[1]{\textcolor{red}{#1}}
\newcommand{\blue}[1]{\textcolor{blue}{#1}}
\newcommand{\green}[1]{\textcolor{green}{#1}}
\newcommand{\magenta}[1]{\textcolor{magenta}{#1}}

\newcommand{\sla}[1]{\not\! #1}
\def\ohalf{{\textstyle{1\over 2}}}
\def\half{{\textstyle{1\over 2}}}
\def\vqhalf{{\textstyle{\vec{Q}\over 2}}}
\def\qhalf{{\textstyle{Q\over 2}}}
\def\osix{{\textstyle{1\over 6}}}
\def\vqsix{{\textstyle{\vec{Q}\over 6}}}
\def\thalf{{\textstyle{3\over 2}}}
\def\fourth{{\textstyle{1\over 4}}}
\def\tfor{{\textstyle{3\over 4}}}

\newcommand{\inp}[2]{\langle{#1}|{#2}\rangle}
\newcommand{\vev}[2]{\langle{#1}\rangle}
\newcommand{\slas}[1]{\not\!{#1}}
\def\ohalf{{\textstyle{1\over 2}}}
\def\thalf{{\textstyle{3\over 2}}}
\def\fhalf{{\textstyle{5\over 2}}}
\def\shalf{{\textstyle{7\over 2}}}
\def\nhalf{{\textstyle{9\over 2}}}


\newcommand\Tr{\,{\rm Tr}\,}
\newcommand\re{\Re\mbox{e}}
\newcommand\im{\Im\mbox{m}}
\newcommand{\beq}{\begin{equation}}
\newcommand{\eeq}{\end{equation}}

\pagestyle{fancy}
\lhead{\scriptsize{ÉCOLE POLYTECHNIQUE FÉDÉRALE DE LAUSANNE\\
3ème année de Physique\\
Baptiste CLAUDON}}
\rhead{\empty}
\renewcommand{\headrulewidth}{0pt}

\theoremstyle{definition}
\newtheorem{Def}{Definition}
\newcommand{\begdef}{\begin{Def}}
%\newcommand{\enddef}{\end{Def}}
\newtheorem{Prop}{Proposition}
\newcommand{\begpro}{\begin{Prop}}
%\newcommand{\endpro}{\end{Prop}}
\newtheorem{The}{Theorem}
\newcommand{\begthe}{\begin{The}}
%\newcommand{\endthe}{\end{The}}
\newtheorem*{Pre}{Proof}
\newtheorem{Ex}{Example}
\newtheorem{Post}{Postulat}

\begin{document}

\begin{titlepage}
   \vspace*{\stretch{1.0}}
   \begin{center}
      \Huge\textbf{Introduction to\\Partial Differential Equations}\\
      %\Large{Introduction aux Ondes\\ et à la Mécanique Quantique}\\ 
      \text{ }\newline
      \large{Baptiste Claudon}\\
      \today
   \end{center}
   \vspace*{\stretch{2.0}}
\end{titlepage}

\tableofcontents\newpage
\newpage
\section{Definition and Classification of PDE}
\subsection{Notation and definitions}
\begin{Def}
Given a domain $\Omega\subset\mathbb R^n$ and a smooth function $u:\bar\Omega\to\mathbb R$, denote the tensor of all partial derivatives of order $k$ by:
\beq
D^k u=\left(\frac{\partial^k u}{\partial x_{i_1}...\partial x_{i_k}}\right)_{(i_1,...,i_k)\in{1,...,n}^k}
\eeq
\end{Def}
\begdef
A $k$-th order PDE on a domain $\Omega$ is an expression of the form:
\beq
F(D^k u(x),...,u(x),x)=0), \forall x\in\Omega
\label{pde}
\eeq
where $F:\mathbb R^{n^k}\times...\times \mathbb R\times\Omega \mapsto\mathbb R$ is given and $u:\Omega\mathbb R$ is the unknown.
\end{Def}
\begdef
We call a solution to \ref{pde} a classical solution.
\end{Def}
\begin{Def}
A partial differential equation is called linear if \ref{pde} is of the form:
\beq
a_{i_1...i_s}(x)\frac{\partial^su(x)}{\partial x_{i_1}...\partial x_{i_s}}=f(x)
\eeq
It is called semi-linear if:
\beq
F(D^k u(x),...,u(x),x)=a_{i_1...i_k}(x)\frac{\partial^ku(x)}{\partial x_{i_1}...\partial x_{i_k}}+\tilde{F}(D^{k-1} u(x),...,u(x),x)
\eeq
It is called quasi-linear if:
\beq
F(D^k u(x),...,u(x),x)=a_{i_1...i_k}(D^{k-1} u(x),...,u(x),x)\frac{\partial^ku(x)}{\partial x_{i_1}...\partial x_{i_k}}+\tilde{F}(D^{k-1} u(x),...,u(x),x)
\eeq
If none of these cases happens, the PDE is called fully non linear.
\end{Def}
\subsection{Linear Second Order PDE}
Since for a classical solution to this problem, partial derivatives commute, one can assume the matrix $A$ of highest order coefficient is symmetrical.
\begdef
A linear second order PDE is called elliptic at $x\in\Omega$ if $A(x)$ is positive (or negative) definite. It is parabolic if it has one eigenvalue zero and all other positive (or negative). It is hyperbolic if it has one negative (resp. positive) eigenvalue and all other positive (resp. negative). If the property is satisfied on the whole domain $\Omega$, we say it is satisfied uniformly.
\end{Def}


\newpage
\appendix
\section{Gauss-Green formulae and integration by parts}
Those formulae follow directly from vector identities and the divergence theorem.
Let $\Omega\subset\mathbb R^n$ be a bounded domain with $C^1$ boundary.
\begpro\textbf{Green-Gauss formula}
\beq
\forall u\in C^1(\overline{\Omega}):\int_\Omega\partial_{x_i}udx=\int_{\partial\Omega}u\nu_idS, i=1,...,n
\eeq
\end{Prop}
\begpro\textbf{Integration by parts formula}
\beq
\forall u,v\in C^1(\overline{\Omega}):\int_\Omega\partial_{x_i}uvdx=\int_\Omega\partial_{x_i}vudx+\int_{\partial\Omega}uv\nu_idS, i=1,...,n
\eeq
\end{Prop}
\begpro\textbf{First Greens' identity}
\beq
\forall u\in C^2(\overline{\Omega}) : \int_\Omega\nabla^2udx=\int_{\partial\Omega}\partial_\nu udS
\eeq
\end{Prop}
\begpro\textbf{Second Greens' identity}
\begin{eqnarray}
\forall u,v\in C^2(\overline{\Omega}) : \int_\Omega\nabla^2uvdx=-\int_{\Omega}\nabla u\cdot\nabla vdx+\int_{\partial\Omega}\partial_\nu uvdS\\=\int_{\Omega}u\nabla^2 vdx+\int_{\partial\Omega}\partial_\nu uv-u\partial_\nu vdS
\end{eqnarray}
\end{Prop}




\end{document}

