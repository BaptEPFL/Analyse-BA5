\part{Espaces Fonctionnels}
\section{Le théorème de Stone-Weierstrass}
\begthe\textbf{Théorème de Diniz}
Soit $(f_n)$ une suite de fonctions réelles et continues définies sur un compact $K\subset\mathbb R^n$ et convergent simplement et de manière monotone vers $f\in C(K,\mathbb R)$. Alors cette suite converge uniformément vers $f$.
\end{The}
\begpre
Choisir, sans perte de généralité que $(f_n)$ est décroissante et converge simplement vers $0$. Soit $\epsilon>0$. Poser pour $n\in\mathbb N$ : 
\beq
V_n=\{x\in K:f_n(x)\leq\epsilon\}
\eeq
Par continuité des fonctions de la suite, tous ces ensembles sont des ouverts. Puisque la suite tend vers $0$, on a : 
\beq
K=\bigcup_{n\in\mathbb N}V_n=K
\eeq
$K$ étant compact, il existe un nombre $F\in\mathbb N$ tel que
\beq
K=\bigcup_{n=0}^FV_n=K
\eeq
Puisque la suite est monotone décroissante, on a que $m<n$ implique $V_m\subseteq V_n$, donc $V_F=K$.
\qed
\end{Pre}

\begdef
Soit $F$ une famille de fonctions définies sur un ensemble $X\subset\mathbb R^n$. On dit que $F$ sépare $X$ si :
\beq
\forall x,y\in X, x\neq y,\exists f\in F:f(x)\neq f(y)
\eeq
\end{Def}
\begdef
On dit que $F$ ne s'annule pas sur $X$ si :
\beq
\forall x\in X\exists f\in F:f(x)\neq0
\eeq
\end{Def}
\begdef
Si $B$ est un sous-ensemble d'une $\mathbb K$-algèbre $A$, alors la $\mathbb K$-algèbre engendrée par $B$, $\mathcal A_\mathbb K(B)$ est la plus petite $\mathbb K$-algèbre contenant $B$.
\end{Def}
\begthe\textbf{Théorème de Stone-Weierstrass} Soit $X\subset\mathbb R^n$ un ensemble compact et soit $F\subseteq C(X,\mathbb R)$ une famille de fonctions qui sépare $X$ et qui ne s'annule pas sur $X$. Alors l'algèbre réelle $\mathcal A_\mathbb K(F)$ engendrée par $F$ est uniformément dense dans $C(X,\mathbb R)$ : 
\beq
\overline{\mathcal A_\mathbb K(F)}^{||\cdot||_\infty}=C(X,\mathbb R)
\eeq
\label{sw}
\end{The}
La preuve est laissée en exercice.
\begcor
Soit $X\subset\mathbb R^n$ un ensemble compact et $F\subseteq C(X,\mathbb C)$ une famille de fonction qui sépare $X$, invariante sous conjugaison complexe et qui ne s'annule pas sur $X$. Alors l'algèbre complexe $\mathcal A_\mathbb C(F)$ engendrée par $F$ est uniformément dense dans $C(X,\mathbb C)$.
\end{Cor}
\begpre
On a $F=F^*$ car : 
\beq F^*\subseteq F=(F^*)^*\subseteq F^*\eeq Comme $F$ sépare $X$ et ne s'annule pas sur $X$, $G=(F+F^*)\cup i(F-F^*)$ ne s'annule pas sur $X$ non plus et sépare aussi $X$. Or $F\subseteq C(X,\mathbb R)$ et par le théorème de Stone-Weierstrass \ref{sw}, $C(X,\mathbb R)=\overline{\mathcal A_\mathbb R(G)}$. Comme $C(X,\mathbb C)=C(X,\mathbb R)+iC(X,\mathbb R)$ et que $\overline{\mathcal A_\mathbb R(G)},i\overline{\mathcal A_\mathbb R(G)}\subset \overline{\mathcal A_\mathbb C(G)}$, on a que $C(X,\mathbb C)=\overline{\mathcal A_\mathbb C(G)}$. Or : $\mathcal A_\mathbb C(G)=\mathcal A_\mathbb C(F)$.
\qed
\label{cor1}
\end{Pre}

\begcor
Soit $X\subset\mathbb R$ un ensemble compact. L'ensemble $\mathbb C[ X]$ est uniformément dense dans $C(X,\mathbb C)$.
\end{Cor}
\begpro
$\mathbb C[ X]=\mathcal A_\mathbb C(\{1,id_X\})
$ vérifie les hypothèses du corollaire \ref{cor1}.
\qed
\end{Prop}

\begcor
Soit $I=[a,b]$ un intervalle fermé de $\mathbb R$. L'algèbre engendrée sur les complexes par $F$ défini comme : 
\beq
F={e^{2\pi ni\frac{x-a}{x-b}},x\in I, n\in\mathbb N}
\eeq
est uniformément dense dans $V=\{f:f\in C([a,b],\mathbb C),f(a)=f(b)\}$. 
\end{Cor}
\begpro
La fonction $\varphi$ définie par :
\beq
\varphi : [a,b]\to\partial B_1{0}, x\mapsto e^{2\pi ni\frac{x-a}{x-b}}
\eeq
induit un homéomorphisme isométrique $\Phi:C(\partial B_1(0),\mathbb C)\to V, f\mapsto f\circ\varphi$. Or, $C(\partial B_1(0),\mathbb C)=overline{\mathcal A_\mathbb C(\{1,z\mapsto z,z\mapsto z^*\})}$ puisque $\{1,z\mapsto z,z\mapsto z^*\}$ satisfait les hypothèses du corollaire \ref{cor1} et $F=\Phi|_{\mathcal A_\mathbb C(\{1,z\mapsto z,z\mapsto z^*\})}$.
\qed
\end{Prop}

\section{Complétion des espaces $L^p$}
\begdef
Une mesure $\mu$ définie sur une $\sigma$-algèbre $\Sigma\subseteq\mathcal P(\mathbb R^n)$ borélienne est dite intérieurement régulière si :
\beq
\forall E\in\Sigma, \mu(E)=\sup\{\mu(K):K \text{ compact et } K\subset E\}
\eeq
Elle est dite extérieurement-régulière si
\beq
\forall E\in\Sigma, \mu(E)=\inf\{\mu(V):V \text{ ouvert et } V\subset E\}
\eeq
Elle est enfin régulière si elle est simultanément extérieurement et intérieurement régulière, et localement finie si :
\beq
\forall x\in\mathbb R^n,\exists \text{ un ouvert }U\in\Sigma : \mu(U)<\infty
\eeq
\end{Def}

\begin{The}
Soit $\mu$ une mesure régulière et localement finie, $f\in L^p(\mathbb R^n,\mu)$ pour $1\leq p<\infty$ et $\epsilon>0$. Alors :
\beq
\exists \varphi\in C_c(\mathbb R^n):||f-\varphi||_p<\epsilon
\eeq
\end{The}

Preuve laissée en exercice.

\begthe
Si $\mu$ est une mesure régulière et localement finie, l'espace de Hilbert $L^2(\mathbb R^n,\mu)$ est séparable.
\end{The}

\begpre
Soit $f\in C_c(\mathbb R^n)$. Des fonctions qui sont des combinaisons linéaires du type $\sum_{k=1}^m\alpha_k\chi_{E_k}$ pour $E_k=\times_{j=1}^n ]a_j,b_j]$ approchent $f$ uniformément dans $C_c(\mathbb R^n)$. Par conséquent, pour une mesure régulière et localement finie, les fonctions de ce type approchent $f$ dans $L^2(\mathbb R^n,\mu)$. On pourrait donc choisir les $\alpha_k,a_k,b_k$ rationnels pour approcher $f$ par une famille dénombrable.
\qed
\end{Pre}

